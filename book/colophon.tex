\thispagestyle{empty}

\begin{figure}[b]

{\LARGE Algorithm $\mathcal W$入門}

\hrulefill

\begin{tabular}{rl}
2011年12月31日 & 初版第1版発行
\end{tabular}

\vspace{1em}

\begin{tabular}{ll}
著者     & 坂口和彦 \\
サークル & coins-11 \\
印刷所   & ちょ古っ都製本工房 \\%の予定
連絡先   & pi8027@gmail.com \\
\end{tabular}

\hrulefill

\end{figure}

\section*{著者紹介}

\subsection*{坂口 和彦(pi8027)}

筑波大学情報科学類B1の学生。
最も良く使うプログラミング言語はHaskellだが、最近はAgda2やCoqなどの依存型付きの言語に興味がある。

情報特別演習Iという科目の演習でこの本の執筆をしている。
単位が取れるかどうかは、この本が無事出せるかどうかにかかっているらしい。

\begin{description}
\item[website] http://stricter.org/
\item[github] http://github.com/pi8027
\item[twitter] http://twitter.com/pi8027
\end{description}

