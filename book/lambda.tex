
本章では、$\lambda$ 計算の基礎を説明します。
$\lambda$ 計算は、関数の計算に関する理論を扱うための計算体系です。
現在使われている多くの関数型プログラミング言語の基礎には、$\lambda$ 計算があると言えます。

次章から書く型推論の対象となる言語も、
最初は $\lambda$ 計算のシンプルな型システムとして知られる単純型付き $\lambda$ 計算です。

この章では、前提知識として必要な $\lambda$ 計算と単純型付き $\lambda$ 計算について説明します。

\section{関数}

\textbf{関数 (function)}は、与えられた値に依存して値が決定されるような対応を表します。
例えば、与えられた数値に $1$ を足した値を対応させる関数 $s$ は、
\[
  s(x) = x+1
\]
のように表記します。

上のように定義された関数 $s$ に値$2$を与えるには、
\[
  s(2)
\]
のように表記します。また、この値は $2+1$、即ち$3$となります。

さて、このような関数を値として扱うことを考えてみましょう。
関数名のみを書くことで、関数の値を指すものとします。
値は関数に与えられるので、関数も関数に与えられるはずです。
例えば、
\[
  a(f, x) = f(x)
\]
とすると、関数 $f$ と値 $x$ に依存し、
関数 $f$ に値 $x$ を与えた値を対応させる関数 $a$ を書けます。

$\lambda$ 計算では、関数を $\lambda$ という記号を使って
$\lambda \text{変数} . \text{変数に依存する式}$ と表記します。
例えば、関数 $s$ は、
\[
  \lambda x . x + 1
\]
と表記します。\footnote{本来の $\lambda$ 計算には数値や加算は含まれませんが、
分かりやすいように適宜このような表記を使います。}。

関数に値を与えることを、\textbf{関数適用 (function application)}
もしくは単に\textbf{適用 (application)} といいます。
$\lambda$ 計算の関数適用は、関数と引数を順番に並べて表記します。
引数部分を括弧で括る必要はありません。
例えば、関数 $(\lambda x . x + 1)$ に値 $2$ を与えるには、
\[
  (\lambda x . x + 1) \, 2
\]
と表記します。左が関数、右が引数を表しています。

関数を値として扱うような関数も問題なく扱えます。例えば、関数適用の関数は、
\[
  \lambda f . \lambda x . f \, x
\]
と書けます。

この関数は元々2引数の関数ですが、関数を返す関数のように書かれています。
これは、2引数の関数を1番目の引数を取って残りの1引数関数が求まる関数に変換しています。
3引数の関数 $f(x, y, z)$ も、$\lambda x . \lambda y . \lambda z . f(x, y, z)$
のように変換できます。4引数以上でも、同様の変換を適用できます。

このように、関数を結果とする関数によって表現された多変数関数を
\textbf{カリー化された関数 (curried function)} といいます。
カリー化によって、どのような多変数関数も複数の1引数関数で表現できます。

これらの記号 $\lambda$ で関数を表すような計算の表現を
$\lambda$ 計算の\textbf{項 (term)} もしくは\textbf{$\lambda$ 項 ($\lambda$-term)} と呼びます。

\section{$\lambda$ 計算}

この節では、$\lambda$ 項を定義します。

変数の集合は、可算無限集合であるものとします。
即ち、変数の集合は自然数と一対一の対応を持ちます。
変数を表すメタ変数としては、$x, y, z, \dots$ などを用います。

$\lambda$ 計算の項を、図\ref{fig:lambda-term}のように帰納的に定義します。
\begin{figure}[htbp]
  \begin{align*}
    e & \bnfcce  x             && \text{(変数)} \\
      & \bnfvert e \, e        && \text{(関数適用)} \\
      & \bnfvert \lambda x . e && \text{($\lambda$ 抽象)}
  \end{align*}
  \caption{$\lambda$ 計算の項}
  \label{fig:lambda-term}
\end{figure}

このような表現の形式を、\textbf{BNF (Backus-Naur form)} といいます。
この表記では、構造の名前、$::=$、中身として取り得る表現の列を $|$ で区切った物を並べて書きます。
BNF は良く構文の表現に利用されますが、本書では主に構造の表現に利用しています。

図\ref{fig:lambda-term}の定義を分かりやすく書き直すと、以下のような意味になります。

\begin{itemize}
  \item 変数 $x$ は $\lambda$ 項である。但し、$x$ は任意の変数である。
  \item 関数適用 $e \, e$ は $\lambda$ 項である。
        但し、2つの $e$ はどちらも任意の $\lambda$ 項である。
  \item $\lambda$ 抽象 $\lambda x . e$ は $\lambda$ 項である。
        但し、$x$ は任意の変数であり、$e$ は任意の $\lambda$ 項である。
  \item 以上の規則で作れる $\lambda$ 項のみが、正しい $\lambda$ 項である。
\end{itemize}

項の定義に $\lambda$ 抽象が含まれていますが、これは前節までの関数に対応するものです。

項のことを\textbf{式}という場合もあります。

また、図\ref{fig:lambda-term}の定義を見て分かる通り、
純粋な $\lambda$ 計算には整数値やブール値などの定数が存在しません。
しかし、整数値やブール値などを関数を使ってエンコードできることが知られています。
ここでは分かりやすいように適宜整数などを使うことがあります。

適切に構造を表現するために全ての項の外側に括弧を付けて項を表記しますが、
表記を簡単にするために幾つかの省略のルールがあります。

$(e_1 \, e_2) \, e_3$ のような関数適用は $e_1 \, e_2 \, e_3$ と省略できます。
すなわち、関数適用の構文は左結合です。

これによって、カリー化された多変数関数の適用 $(\dots((f \, x_1) \, x_2) \dots x_n)$
は $f \, x_1 \, x_2 \dots x_n$ のように省略できます。

$\lambda x. (e_1 \, e_2)$ のような $\lambda$ 抽象は $\lambda x. e_1 \, e_2$ と省略できます。
すなわち、関数適用の方が $\lambda$ 抽象に比べて優先されます。

$\lambda$ 項 $e$ の一部分となっているような項を、$e$ の
\textbf{$\lambda$ 部分項 ($\lambda$-subterm)} もしくは単に\textbf{部分項 (subterm)} と呼びます。

$\lambda$ 項において、変数 $x$ の出現とは、項の上で変数 $x$ が現れる位置を意味します。
とある変数 $x$ の出現が $\lambda$ 抽象 $\lambda x . e$ の部分項であるとき、
変数 $x$ の出現は\textbf{束縛されている (bound)} といい、
そうでない変数 $x$ の出現は\textbf{自由 (free)} であるといいます。

束縛された出現を持つ変数を\textbf{束縛変数 (bound variable)} といい、
自由な出現を持つ変数を \textbf{自由変数 (free variable)} といいます。

\begin{exercise}

次のうち、$\lambda$ 計算の項として正しくない表現はどれか。

\begin{enumerate}
  \item $\lambda x . x$
  \item $\lambda f . f \, x$
  \item $f \, (\lambda x)$
\end{enumerate}

\subparagraph{解答}

\begin{enumerate}
  \item $x$ が変数として正しく、$\lambda x . x$ 全体も $\lambda$ 抽象として正しいので
        正しい項になっています。
  \item $f$ と $x$ がそれぞれ変数として正しく、$f \, x$ が関数適用として正しく、
        $\lambda f . f \, x$ 全体も $\lambda$ 抽象として正しいので正しい項になっています。
  \item $f$ と $x$ はそれぞれ変数として正しい形になっていますが、
        $\lambda x$ は正しい $\lambda$ 計算の項の形をしていません。
        よって、$f \, (\lambda x)$ 全体も正しい $\lambda$ 計算の項ではありません。
\end{enumerate}

$\lambda$ 計算の項として正しくない表現は、三番目の $f \, (\lambda x)$ です。

\end{exercise}

\begin{exercise}

次の $\lambda$ 項の下線部の変数は、束縛変数と自由変数のどちらか。

\begin{enumerate}
  \item $\lambda x . \lambda y . \underline{x}$
  \item $\lambda y . (\lambda f . f \, x) \, \underline{f}$
\end{enumerate}

\subparagraph{解答}

\begin{enumerate}
  \item 下線部の変数 $x$ の出現が $\lambda x . \lambda y . x$ の部分項となっているので、束縛変数です。
  \item 下線部の変数 $f$ の出現は $\lambda f . e$ の形の項の部分項となっていないので、自由変数です。
        $\lambda f . f \, x$ という部分項がありますが、
        下線分の変数 $f$ の出現がこの項の部分項になっていないことに注意してください。
\end{enumerate}

\end{exercise}

\section{$\alpha$ 同値}

\textbf{$\alpha$ 同値 ($\alpha$-equivalent)}は、$\lambda$ 項の構文上の同値関係です。
項 $e_1$ と項 $e_2$ が $\alpha$ 同値であるということを記号 $\alphaeq$ を用いて
$e_1 \alphaeq e_2$ と表記します。

$\alpha$ 同値の基本は、とある項の束縛変数の名前を置き換えても、
それは同じ項であるということです。例えば、$\lambda x . x$ は $\lambda y . y$ と $\alpha$ 同値です。

この関係を定義するためには、置き換える対象の束縛変数を束縛している $\lambda$ 抽象
$\lambda x . e$ に関して、項 $e$ に自由出現する変数 $x$ を全て置き換えるた項を考える必要があります。

まず、この置き換えについて考えましょう。項に自由出現する特定の変数を別の項で置き換える操作を
\textbf{代入 (substitution)} と呼びます。

自由変数 $x$ を全て項 $e$ に置き換える代入を
\[
  [x := e]
\]
と表記します。

項 $e_1$ に代入 $[x := e_2]$ を適用した項、即ち項 $e_1$ 中の自由変数 $x$ を全て項 $e_2$ に置き換える
ことによって得られる項を
\[
  e_1 [x := e_2]
\]
と表記します。

代入の適用は、図 \ref{fig:lambda-substitute} のようにも定義できます。

\begin{figure}[htbp]
  \begin{align*}
    y [x := e_1] & = \left \{
      \begin{array}{ll}
        e_1 & (x = y) \\
        y & (x \neq y)
      \end{array}
      \right. \\
    e_2 \, e_3 [x := e_1] & = (e_2 [x := e_1]) \, (e_3 [x := e_1]) \\
    (\lambda y . e_2) [x := e_1] & = \left \{
      \begin{array}{ll}
        \lambda y . (e_2 [x := e_1]) & (x \neq y) \\
        \lambda y . e_2 & (x = y)
      \end{array}
      \right.
  \end{align*}
  \caption{代入}
  \label{fig:lambda-substitute}
\end{figure}

$\alpha$ 同値の関係を、図\ref{fig:alpha-equivalent}のように定義します。

\begin{figure}[htbp]
  \[
    \infere{Eq-Alpha}{
      (\lambda x . e) \alphaeq (\lambda y . e [x := y])
    }{
      \text{項 $e$ に変数 $y$ が自由出現しない}
    }
  \]
  \[
    \infere{Eq-Refl}{
      e \alphaeq e
    }{}
  \]
  \[
    \infere{Eq-Sym}{
      e' \alphaeq e
    }{
      e \alphaeq e'
    }
  \]
  \[
    \infere{Eq-Trans}{
      e_1 \alphaeq e_3
    }{
      e_1 \alphaeq e_2 & e_2 \alphaeq e_3
    }
  \]
  \[
    \infere{Eq-App}{
      e_1 \, e_2 \alphaeq e_1' \, e_2'
    }{
      e_1 \alphaeq e_1' & e_2 \alphaeq e_2'
    }
  \]
  \[
    \infere{Eq-Abs}{
      \lambda x . e \alphaeq \lambda x . e'
    }{
      e \alphaeq e'
    }
  \]
  \caption{$\alpha$ 同値}
  \label{fig:alpha-equivalent}
\end{figure}

この分数のような表記は、上が仮定で下が結論であるような規則を表しています。
つまり、上に書いてあることが成り立てば、下に書いてあることを成り立つとして良いということになります。

規則の右側に書かれている \textsc{Eq-Alpha}, \textsc{Eq-Refl}, \textsc{Eq-Sym}, \textsc{Eq-Trans},
\textsc{Eq-App}, \textsc{Eq-Abs} は、規則の名前を表しています。

規則を1つずつ見てみましょう。

規則 \textsc{Eq-Alpha} は、束縛変数の置き換えの規則です。
中身は、項 $e$ に変数 $y$ が自由出現しないなら、項 $\lambda x . e$ と $\lambda x . e [x := y]$ は
$\alpha$ 同値であるということを表しています。
項 $e$ に変数 $y$ が自由出現してはいけないのは、そのような変数に置き換えてしまうと
自由出現していた変数がその $\lambda$ 抽象で束縛されてしまい、意味が変わってしまうからです。

規則 \textsc{Eq-Refl}, \textsc{Eq-Sym}, \textsc{Eq-Trans} は同値関係としての性質、
それぞれ反射性、対象性、推移性を表しています。

規則 \textsc{Eq-App} は、関数適用の左辺と右辺をそれぞれ $\alpha$ 同値な項に置き換えても、
全体として $\alpha$ 同値であるという規則です。

規則 \textsc{Eq-Abs} は、$\lambda$ 抽象の内側の項を $\alpha$ 同値な項に置き換えても、
全体として $\alpha$ 同値であるという規則です。

本書ではこれ以降 $\alpha$ 同値性を構文的な等しさとして扱い、
$=_\alpha$ も紛らわしいなどの問題がなければ単に $=$ と表記します。

\section{変数規約}



\section{$\beta$ 簡約}

$\lambda$ 計算の実行について考えてみましょう。

$\lambda$ 計算の実行の本質は、左辺が $\lambda$ 抽象になっている関数適用
($(\lambda x . e_1) \, e_2$ の形をしている $\lambda$ 項)を解く操作です。
この操作によって得られる項は、項 $e_1$ 中の自由変数 $x$ に項 $e_2$ を代入した項です。

項の中の関数適用を一箇所解く操作を、\textbf{$\beta$簡約 ($\beta$-reduction)} と呼びます。
$e_1$ を $\beta$ 簡約して $e_2$ にできるという二項関係を、
記号 $\betared$ を用いて $e_1 \betared e_2$ と表記します。
$\lambda$ 計算の $\beta$ 簡約を、図 \ref{fig:beta-reduction} のように定義します。

\begin{figure}[htbp]
  \[
    \infere{R-Beta}{
      (\lambda x . e_1) \, e_2 \betared e_1 [x := e_2]
    }{}
  \]
  \[
    \infere{R-App1}{
      e_1 \, e_2 \betared e_1' \, e_2
    }{
      e_1 \betared e_1'
    }
  \]
  \[
    \infere{R-App2}{
      e_1 \, e_2 \betared e_1 \, e_2'
    }{
      e_2 \betared e_2'
    }
  \]
  \[
    \infere{R-Abs}{
      \lambda x . e \betared \lambda x . e'
    }{
      e \betared e'
    }
  \]
  \caption{$\beta$ 簡約}
  \label{fig:beta-reduction}
\end{figure}

\textsc{R-Beta} 規則のように上が空になっている規則は、
無条件で下に書いてあることを認めて良いということを表しています。

$\lambda$ 項の評価は $\beta$ 簡約の並びによって表現できます。

\begin{exercise}

以下の $\beta$ 簡約が正しいことを示せ。

\[
  (\lambda x . x) \, f \, y \betared f \, y
\]

\subparagraph{解答}

これは、$(\lambda x . x) \, f$ の部分を簡約すれば良さそうです。$x [x := f] = f$ なので、
\[
  \infere{R-Beta}{
    (\lambda x . x) \, f \betared f
  }{}
\]
とできます。この部分式は全体から見ると関数適用の左辺なので、
\[
  \infere{R-App1}{
    (\lambda x . x) \, f \, y \betared f \, y
  }{
    (\lambda x . x) \, f \betared f
  }
\]
とできます。
このようにして、$(\lambda x . x) \, f \, y \betared f \, y$ が正しい $\beta$ 簡約であることが示せます。

また、実際はこの2つの証明の段階を繋げて、
\[
  \infere{R-App1}{
    (\lambda x . x) \, f \, y \betared f \,
 y
  }{
    \infere{R-Beta}{
      (\lambda x . x) \, f \betared f
    }{}
  }
\]
のように表記します。
これによって、複雑な構造を持つ証明を分かりやすく記述することができます。

\end{exercise}

\begin{exercise}

以下の $\lambda$ 項を簡約せよ。

\[
  (\lambda x . x \, x) \, (\lambda x . x \, x)
\]

\subparagraph{解答}

外側の関数適用が簡約できます。
$x \, x [x := \lambda x . x \, x] = (\lambda x . x \, x) \, (\lambda x . x \, x)$ なので、
\[
  \infere{R-Beta}{
    (\lambda x . x \, x) \, (\lambda x . x \, x) \betared
    (\lambda x . x \, x) \, (\lambda x . x \, x)
  }{}
\]
となります。

しかし、簡約した結果として自分自身と同じ項が作れてしまいました。
このことから、この項は無限長の $\beta$ 簡約列を作ることができ、評価が停止しないと分かります。

この例以外にも、無限長の $\beta$ 簡約列を作れる $\lambda$ 項は無数に存在しています。

\end{exercise}

\section{型付き $\lambda$ 計算}

前の練習問題のように、無限長の $\beta$ 簡約列を作ることができるような $\lambda$ 項を作ることができます。

また、純粋な $\lambda$ 計算の範囲の話ではありませんが、定数値や定数値に対する演算を含む計算体系では、
特定の種類の定数値のみを対象とできる演算を考えたい場合があります。例えば、
\[
  \text{true} + 1
\]
は加算が数値に対する二項演算であって欲しいとすると、
真偽値 true と整数値 $1$ に対して適用されているため意図に反した計算であると言えます。

このような、停止しない計算や意図に反する計算を「正しくない計算」と見做し、正しくない計算を
表している項を排除する仕組みが\textbf{型付き $\lambda$ 計算 (typed lambda calculus)} です。
\footnote{
ただし、型付き $\lambda$ 計算から派生したプログラミング言語のほとんどは
プログラムが停止しないことや失敗することを許容しています。
しかし、その多くは型によってプログラムの間違いのほとんどを検出し、排除することに成功しています。}

型付き $\lambda$ 計算における\textbf{型 (type)} とは、とある項に対して適用可能な操作の表現です。

\subsection{単純型付き $\lambda$ 計算}

\textbf{単純型付き $\lambda$ 計算 (simply typed lambda calculus, $\lambda^\to$)} は、
型付きラムダ計算の体系の1つです。

単純型付き $\lambda$ 計算の型を、図\ref{fig:stlc-type}のように帰納的に定義します。

\begin{figure}[htbp]
  \begin{align*}
    t & \bnfcce  \alpha  && \text{(型変数)} \\
      & \bnfvert t \to t && \text{(関数の型)}
  \end{align*}
  \caption{単純型付き $\lambda$ 計算の型}
  \label{fig:stlc-type}
\end{figure}

このように、型変数と関数の型だけからなるものを単純型付き $\lambda$ 計算の型とします。
型は同じ型変数を使うことで同じ型であることを表し、
$\alpha \to \beta$ は型 $\alpha$ の値から型 $\beta$ の値への関数を表します。

例えば、$\lambda x. x$ (恒等関数)は $\alpha \to \alpha$ という型を持ちます。

これまでと同様、型の構造を表すために適宜括弧を使うことがあります。
ただし、関数の型は右結合であり、
$t_1 \to (t_2 \to t_3)$ は $t_1 \to t_2 \to t_3$ のように省略できます。

\subsection{型判定}

単純型付き $\lambda$ 計算では、
項に正しく型付けできるということを\textbf{型判定 (type judgement)} で表します。

型環境 $\Gamma$ の下で項 $e$ の型が $t$ である、ということを
\[ \Gamma \vdash e : t \]
と表記します。

型環境 $\Gamma$ は、変数名と型のペアの集合($x_1 : t_1, x_2 : t_2, \dots, x_n : t_n$)です。
これは型を決定する上での文脈であり、変数の型はここから引いてくることになります。

型判定を型環境、項、型の間の関係とみなし、\textbf{型付け関係 (typing relation)} と呼ぶこともあります。

型環境 $\Gamma$ に「変数 $x$ の型が $t$ である」
という情報が含まれていることを $x : t \in \Gamma$ と表記します。
また、空の型環境を $\emptyset$ と表記し、型環境 $\Gamma$ に
「変数 $y$ の型が $t$ である」ということを追加した環境を $\Gamma , y : t$ と表記します。
型環境に関する規則を、図\ref{fig:stlc-type-environment}のように定義します。

\begin{figure}[htbp]
  \[
    \infere{T-Env1}{
      x : t \in \Gamma , x : t
    }{}
  \]
  \[
    \infere{T-Env2}{
      x : t_1 \in \Gamma , y : t_2
    }{
      x \neq y && x : t_1 \in \Gamma
    }
  \]
  \caption{型環境}
  \label{fig:stlc-type-environment}
\end{figure}

規則 \textsc{T-Env1} は、
型環境 $\Gamma, x : t$ に「変数 $x$ の型が $t$ である」が無条件に含まれる、と読みます。

規則 \textsc{T-Env2} は、
型環境 $\Gamma$ に「変数 $x$ の型が $t_1$ である」が含まれていてかつ $x \neq y$ であれば、
型環境 $\Gamma, y : t_2$ に「変数 $x$ の型が $t_1$ である」が含まれる、と読みます。

これらの規則は、単純に型環境に特定の変数と型のペアが含まれているということを
表す規則でないということに注意してください。
例えば、$x : t_1 \in \emptyset, x : t_1, y : t_2$ は成立しますが、
$x : t_2 \in \emptyset, x : t_1, x : t_2$ は成立しません。

なぜこうなっているのかというと、$\lambda x . \lambda x . x$ などの項における末尾の変数 $x$ は、
内側の $\lambda$ 抽象の $x$ であって外側の $\lambda$ 抽象の $x$ ではないからです。
型環境は後から追加された情報ほど深いスコープの変数になっているので、
最後に追加された同じ名前の変数のみを取れるような規則となっています。

単純型付き $\lambda$ 計算の型判定を、図\ref{fig:stlc-type-judgement}のように定義します。
このルールは、$\lambda$ 計算の項の種類1つに対してルールが1つであるということに注意してください。

\begin{figure}[htbp]
  \[
    \infere{T-Var}{
      \Gamma \vdash x : t
    }{
      x : t \in \Gamma
    }
  \]
  \[
    \infere{T-App}{
      \Gamma \vdash e_1 \, e_2 : t_2
    }{
      \Gamma \vdash e_1 : t_1 \to t_2 &
      \Gamma \vdash e_2 : t_1
    }
  \]
  \[
    \infere{T-Abs}{
      \Gamma \vdash \lambda x . e : t_1 \to t_2
    }
    {
      \Gamma, x : t_1 \vdash e : t_2
    }
  \]
  \caption{単純型付き $\lambda$ 計算の型判定}
  \label{fig:stlc-type-judgement}
\end{figure}

規則 \textsc{T-Var} は、型環境 $\Gamma$ に「変数 $x$ の型が $t$ である」が含まれていれば、
型環境 $\Gamma$ の下で式 $x$ の型は $t$ である、と読みます。

規則 \textsc{T-App} は、
型環境 $\Gamma$ の下で式 $e_1$ の型が $t_1 \to t_2$ かつ $e_2$ の型が $t_1$ であれば、
型環境 $\Gamma$ の下で式 $e_1 ~ e_2$ の型は $t_2$ である、と読みます。

規則 \textsc{T-Abs} は、
型環境 $\Gamma , x : t_1$ の下で式 $e$ の型が $t_2$ であれば、
型環境 $\Gamma$ の下で式 $\lambda x . e$ の型が $t_1 \to t_2$ である、と読みます。

また、定数などを含む $\lambda$ 項に型を付ける場合には、
それらの定数や関連する演算に関しての規則を追加する必要があります。
例えば、ブール値(bool 型)と整数値(int 型)と整数値の加算を加えた定義は、
項にブール値と整数値を加え、型に bool 型と int 型を加えた上で、
図\ref{fig:stlc-type-judgement-constants}の型判定規則を追加します。

\begin{figure}[htbp]
  \[
    \infere{T-Bool}{
      \Gamma \vdash c : \text{bool}
    }{
      c \text{はブール値}
    }
  \]
  \[
    \infere{T-Int}{
      \Gamma \vdash c : \text{int}
    }{
      c \text{は整数値}
    }
  \]
  \[
    \infere{T-Plus}{
      \Gamma \vdash e_1 + e_2 : \text{int}
    }{
      \Gamma \vdash e_1 : \text{int} &
      \Gamma \vdash e_2 : \text{int}
    }
  \]
  \caption{定数に関する追加の型判定規則}
  \label{fig:stlc-type-judgement-constants}
\end{figure}

これらの規則は簡単なので、練習も兼ねて自分で読んでみてください。

\subsection{型を付ける}

ここまでに出てきた規則を利用して、実際に $\lambda$ 項に型を付けてみましょう。

\subsubsection{例1 : $\lambda x . x + 1$}

例として、項 $\lambda x . x + 1$ に型を付けてみます。
項の外側から、即ち型判定木の下の方から構築していきましょう。一番外側は $\lambda$ 抽象なので、
\[
  \infere{T-Abs}{
    \emptyset \vdash \lambda x . x + 1 : ?_0 \to ?_1
  }{
    \emptyset , x : ?_0 \vdash x + 1 : ?_1
  }
\]
と書けます。

ただし、ここでの記号 $?$ はまだ確定していない型を表現しています。
同じ番号を振ることで同じ型であることを表現しています。

次に出てきたのは加算なので、
\[
  \infere{T-Abs}{
    \emptyset \vdash \lambda x . x + 1 : ?_0 \to \text{int}
  }{
    \infere{T-Plus}{
      \emptyset , x : ?_0 \vdash x + 1 : \text{int}
    }{
      \emptyset , x : ?_0 \vdash x : \text{int} &
      \emptyset , x : ?_0 \vdash 1 : \text{int}
    }
  }
\]
と書けます。規則 \textsc{T-Plus} によって $?_1$ が int 型に置き換えられたことに注意してください。

次に、左側の項 $x$ の型判定を埋めましょう。規則 \textsc{T-Var} と規則 \textsc{T-Env1} を使うと、
\[
  \infere{T-Abs}{
    \emptyset \vdash \lambda x . x + 1 : \text{int} \to \text{int}
  }{
    \infere{T-Plus}{
      \emptyset , x : \text{int} \vdash x + 1 : \text{int}
    }{
      \infere{T-Var}{
        \emptyset , x : \text{int} \vdash x : \text{int}
      }{
        \infere{T-Env1}{
          x : \text{int} \in \emptyset , x : \text{int}
           }{}
      } &
      \emptyset , x : \text{int} \vdash 1 : \text{int}
    }
  }
\]
と書けます。ここでも、規則 \textsc{T-Env1} の制約により $?_0$ が int 型に置き換わりました。

最後に、定数1の型判定を埋めます。規則 \textsc{T-Int} を使って、
\[
  \infere{T-Abs}{
    \emptyset \vdash \lambda x . x + 1 : \text{int} \to \text{int}
  }{
    \infere{T-Plus}{
      \emptyset , x : \text{int} \vdash x + 1 : \text{int}
    }{
      \infere{T-Var}{
        \emptyset , x : \text{int} \vdash x : \text{int}
      }{
        \infere{T-Env1}{
          x : \text{int} \in \emptyset , x : \text{int}
           }{}
      } &
      \infere{T-Int}{
        \emptyset , x : \text{int} \vdash 1 : \text{int}
      }{
        1 \text{は整数値}
      }
    }
  }
\]
と書けます。これで型判定全体が作れました。

このようにして、
$\lambda x . x + 1$ の型が $\text{int} \to \text{int}$ であることの理由付けができます。

\subsubsection{例2 : $\lambda x . \lambda y . x$}

もう1つの例として、項 $\lambda x . \lambda y . x$ に型を付けてみます。
まず、項全体は $\lambda$ 抽象なので、
\[
  \infere{T-Abs}{
    \emptyset \vdash \lambda x . \lambda y . x : ?_0 \to ?_1
  }{
    \emptyset , x : ?_0 \vdash \lambda y . x : ?_1
  }
\]
と書けます。

次に現れているのも $\lambda$ 抽象なので、
\[
  \infere{T-Abs}{
    \emptyset \vdash \lambda x . \lambda y . x : ?_0 \to ?_2 \to ?_3
  }{
    \infere{T-Abs}{
      \emptyset , x : ?_0 \vdash \lambda y . x : ?_2 \to ?_3
    }{
      \emptyset , x : ?_0 , y : ?_2 \vdash x : ?_3
    }
  }
\]
と書けます。規則 \textsc{T-Abs} の制約により、$?_1$ が $?_2 \to ?_3$ に置き換わりました。

次に現れているのは変数なので、
\[
  \infere{T-Abs}{
    \emptyset \vdash \lambda x . \lambda y . x : ?_0 \to ?_2 \to ?_0
  }{
    \infere{T-Abs}{
      \emptyset , x : ?_0 \vdash \lambda y . x : ?_2 \to ?_0
    }{
      \infere{T-Var}{
          \emptyset , x : ?_0 , y : ?_2 \vdash x : ?_0
      }{
        \infere{T-Env2}{
          x : ?_0 \in \emptyset , x : ?_0 , y : ?_2
         }{
          x \neq y &
          \infere{T-Env1}{
            x : ?_0 \in \emptyset , x : ?_0
          }{}
        }
      }
    }
  }
\]
と書けます。規則 \textsc{T-Env} の制約により $?_0$ と $?_3$ が同一であると分かったため、
$?_3$ を $?_0$ で置き換えました。ただし、これは $?_0$ を $?_3$ で置き換えても構いません。

これで一応型は付けられたように見えますが、記号 $?$ がそのまま残ってしまいました。
これらは、とりあえず型変数で表現することにします。
今の例であれば、$\alpha \to \beta \to \alpha$ のように書きます。

この推論後に残ってしまう $?$ については、また3章で別の形で出てきます。

\begin{exercise}

以下の項が単純型付き $\lambda$ 計算において型付け不可能であることを示せ。

\[
  \text{true} + 1
\]

\subparagraph{解答}

これは節の最初に出てきた意図に反する項の例です。このような項は型システムによって排除されるはずです。
本当に型付け不可能なのか確かめてみましょう。

まず、項全体は加算になっているので、
\[
  \infere{T-Plus}{
    \emptyset \vdash \text{true} + 1 : \text{int}
  }{
    \emptyset \vdash \text{true} : \text{int} &
    \emptyset \vdash 1 : \text{int}
  }
\]
と書けます。

しかし、ここで $\emptyset \vdash \text{true} : \text{int}$ の部分に注目すると、
このような型判定を作れる規則は存在しません。

\end{exercise}

\begin{exercise}

以下の項が単純型付き $\lambda$ 計算において型付け不可能であることを示せ。

\[
  (\lambda x . x \, x) \, (\lambda x . x \, x)
\]

\subparagraph{解答}

これは無限長の $\beta$ 簡約列を作ることができるような $\lambda$ 項の例です。
本当に型付け不可能なのか確かめてみましょう。

まず、項全体が関数適用になっているので、
\[
  \infere{T-App}{
    \emptyset \vdash (\lambda x . x \, x) \, (\lambda x . x \, x) : ?_1
  }{
    \emptyset \vdash \lambda x . x \, x : ?_0 \to ?_1 &
    \emptyset \vdash \lambda x . x \, x : ?_0
  }
\]
と書けます。

次に、左側の $\lambda$ 抽象に注目すると、
\[
  \infere{T-App}{
    \emptyset \vdash (\lambda x . x \, x) \, (\lambda x . x \, x) : ?_1
  }{
    \infere{T-Abs}{
      \emptyset \vdash \lambda x . x \, x : ?_0 \to ?_1
    }{
      \{x : ?_0\} \vdash x \, x : ?_1
    } &
    \emptyset \vdash \lambda x . x \, x : ?_0
  }
\]
と書けます。ここで、$\{x : ?_0\}$ というのは $\emptyset, x : ?_0$ の簡略表記です。
これ以降も、$\emptyset, x_1 : t_1, \dots, x_n : t_n$ を $\{x_1 : t_1, \dots, x_n : t_n\}$
と表記することがあります。

次に、$x \, x$ の関数適用に注目すると、
\[
  \infere{T-App}{
    \emptyset \vdash (\lambda x . x \, x) \, (\lambda x . x \, x) : ?_1
  }{
    \infere{T-Abs}{
      \emptyset \vdash \lambda x . x \, x : ?_0 \to ?_1
    }{
      \infere{T-App}{
        \{x : ?_0\} \vdash x \, x : ?_1
      }{
        \{x : ?_0\} \vdash x : ?_2 \to ?_1 &
        \{x : ?_0\} \vdash x : ?_2
      }
    } &
    \emptyset \vdash \lambda x . x \, x : ?_0
  }
\]
と書けます。

次に、2つの変数 $x$ に注目すると、
\[
  \infere{T-App}{
    \emptyset \vdash (\lambda x . x \, x) \, (\lambda x . x \, x) : ?_1
  }{
    \infere{T-Abs}{
      \emptyset \vdash \lambda x . x \, x : ?_0 \to ?_1
    }{
      \infere{T-App}{
        \{x : ?_0\} \vdash x \, x : ?_1
      }{
          \infere{T-Var}{
          \{x : ?_0\} \vdash x : ?_0 \to ?_1
         }{
          x : ?_0 \to ?_1 \in \{x : ?_0\}
        } &
         \infere{T-Var}{
          \{x : ?_0\} \vdash x : ?_0
        }{
          \infere{T-Env1}{
            x : ?_0 \in \{x : ?_0\}
          }{}
        }
      }
    } &
    \emptyset \vdash \lambda x . x \, x : ?_0
  }
\]
と書けます。規則 \textsc{T-Var} 内の制約により $?_2$ が $?_0$ に置き換わりました。

しかし、ここで $x : ?_0 \to ?_1 \in \{x : ?_0\}$ の部分がまだ書かれていません。
この形を見る限りでは規則 \textsc{T-Env1} が適用されるように見えますが、
もしそうだとすると $?_0 = ?_0 \to ?_1$ であるような型 $?_0$ がなくてはいけません。
しかしそれは $((\dots \to ?_1) \to ?_1) \to ?_1$ のような無限長の型になってしまいます。
\footnote{型推論において、このような無限長の型が作られないことを保証し、
型推論が停止しないことを防ぐための仕組みを出現検査(occurs check)と言います。
これは2章で説明します。}

型の定義は帰納的であるため、このような無限長の型を作ることはできません。
このことから、項 $(\lambda x . x \, x) \, (\lambda x . x \, x)$ が型付けできないことが分かります。

\subparagraph{別解}

帰納法により、「項が型付けできればその項の任意の部分は型付け可能」、
また、その対偶「とある項の部分が型付け不可能であれば全体も型付け不可能」が証明できます。

$\lambda x . x \, x$ 自体が型付け不可能なので、この2つの事実から
$(\lambda x . x \, x) \, (\lambda x . x \, x)$ が型付け不可能であると言えます。

\end{exercise}

\section{型検査}



