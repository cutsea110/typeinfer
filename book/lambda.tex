
$\lambda$ 計算は計算体系の一つで、計算を\textbf{関数}と\textbf{関数適用}で表します。
現在使われている多くの関数型プログラミング言語の基礎には、$\lambda$ 計算があると言えます。

次章から書く型推論の対象となる言語も、
最初は $\lambda$ 計算のシンプルな型システムとして知られる単純型付き $\lambda$ 計算です。

この章では、前提知識として必要な $\lambda$ 計算と単純型付き $\lambda$ 計算について説明します。

\section{$\lambda$ 計算}

$\lambda$ 計算では、$\lambda$ 抽象(関数)を記号 $\lambda$ を使って
$\lambda \text{パラメータ名} . \text{結果}$ と表記します。
例えば、数値を取って1を足して返す関数は、
\[
  \lambda x . x+1
\]
と表記します\footnote{本来の $\lambda$ 計算には数値や加算は含まれませんが、
分かりやすいように適宜このような表記を使います。}。

作った関数を値に適用する仕組みが必要です。関数適用は、関数と引数を順番に並べて表記します。
例えば、引数をそのまま返す関数にその関数自身を適用するには、
\[
  (\lambda x. x) \, (\lambda x. x)
\]
と表記します。左が関数、右が引数を表しています。

このような計算そのものの表現を\textbf{項 (term)}と呼びます。
$\lambda$ 計算の項は、図\ref{fig:lambda-term}のように帰納的に定義されます。

\begin{figure}[htbp]
  \begin{align*}
    e & \bnfcce  x             && \text{(変数)} \\
      & \bnfvert e \, e        && \text{(関数適用)} \\
      & \bnfvert \lambda x . e && \text{($\lambda$ 抽象)}
  \end{align*}
  \caption{$\lambda$ 計算の項}
  \label{fig:lambda-term}
\end{figure}

変数、関数適用、$\lambda$ 抽象からなる表現だけが $\lambda$ 計算の項と呼ばれます。
項のことを\textbf{式}という場合もあります。

また、図\ref{fig:lambda-term}の定義を見て分かる通り、
純粋な $\lambda$ 計算には整数値やブール値などの定数が存在しません。
しかし、整数値やブール値などを関数を使ってエンコードできることが知られています。
ここでは分かりやすいように適宜整数などを使うことがあります。

適切に構造を表現するために括弧を使って項を表記しますが、
表記を簡単にするために幾つかの省略のルールがあります。

$(e_1 \, e_2) \, e_3$ のような関数適用は $e_1 \, e_2 \, e_3$ と省略できます。
すなわち、関数適用の構文は左結合です。

$\lambda x. (e_1 \, e_2)$ のような $\lambda$ 抽象は $\lambda x. e_1 \, e_2$ と省略できます。
すなわち、関数適用の方が $\lambda$ 抽象に比べて優先されます。

\begin{exercise}

次のうち、$\lambda$ 計算の項として正しくない表現はどれか。

\begin{enumerate}
  \item $\lambda x . x$
  \item $\lambda f . f \, x$
  \item $f \, (\lambda x)$
\end{enumerate}

\subparagraph{解答}

1番目は、$x$ が変数として正しく、
$\lambda x . x$ 全体も $\lambda$ 抽象として正しいので正しい項になっています。

2番目は、$f$ と $x$ がそれぞれ変数として正しく、$f \, x$ が関数適用として正しく、
$\lambda f . f \, x$ 全体も $\lambda$ 抽象として正しいので正しい項になっています。

3番目は、$f$ と $x$ はそれぞれ変数として正しい形になっていますが、
$\lambda x$ は正しい $\lambda$ 計算の項の形をしていません。
よって、$f \, (\lambda x)$ 全体も正しい $\lambda$ 計算の項ではありません。

$\lambda$ 計算の項として正しくない表現は、三番目の $f \, (\lambda x)$ です。

\end{exercise}

\begin{exercise}

以下の $\lambda$ 計算の項を実行した結果は何になるか、答えよ。

\begin{enumerate}
  \item $(\lambda x . x) \, (\lambda y . y)$
  \item $(\lambda f . \lambda g . \lambda x . f \, x \, (g \, x)) \,
		(\lambda x . \lambda y . x) \, (\lambda x . \lambda y . x)$
\end{enumerate}

\subparagraph{解答}

1番目は、外側の関数適用を実行すると $\lambda y . y$ となります。よって、実行結果は $\lambda y . y$
となります。

2番目は、まず項
$(\lambda f . \lambda g . \lambda x . f \, x \, (g \, x)) \, (\lambda x . \lambda y . x)$
の関数適用を実行します。結果は、$\lambda g . \lambda x . (\lambda x . \lambda y . x) \, x \, (g \, x)$
となり、これの内側の2つの関数適用を実行すると $\lambda g . \lambda x . x$ となります。
よって、項全体は $(\lambda g . \lambda x . x) \, (\lambda x . \lambda y . x)$ となります。
この項の外側の関数適用を実行すると、$\lambda x . x$ となります。
実行結果は $\lambda x . x$ です。

\end{exercise}

\section{$\lambda$ 計算の実行}

$\lambda$ 計算の実行について考えてみましょう。

前の練習問題で分かる通り、$\lambda$ 計算の実行の本質は、左辺が $\lambda$ 抽象になっている関数適用
($(\lambda x . e_1) \, e_2$ の形をしている $\lambda$ 項)を解く操作です。
この操作は、項 $e_1$ の中の自由変数 $x$ を全て項 $e_2$ で置き換える操作です。
この自由変数を全て置き換える操作を\textbf{代入}と呼び、$[e_2/x] e_1$ と表記します。

図 \ref{fig:lambda-substitute} に代入の定義を示します。

\begin{figure}[htbp]
  \begin{align*}
    [e_1/x] y & = \left \{
      \begin{array}{ll}
        e_1 & (x = y) \\
        y & (x \neq y)
      \end{array}
      \right. \\
    [e_1/x] e_2 \, e_3 & = ([e_1/x] e_2) \, ([e_1/x] e_3) \\
    [e_1/x] \lambda y . e_2 & = \left \{
      \begin{array}{ll}
        \lambda y . [e_1/x] e_2 & (x \neq y) \\
        \lambda y . e_2 & (x = y)
      \end{array}
      \right.
  \end{align*}
  \caption{代入}
  \label{fig:lambda-substitute}
\end{figure}

次に、代入を使って $\lambda$ 計算の項の関数適用を評価するを作りましょう。

項の中の関数適用を一度実行する事を、\textbf{$\beta$簡約 ($\beta$-retuction)}と呼びます。
$e_1$ を $\beta$ 簡約して $e_2$ にできるという事を、
記号 $\betared$ を用いて $e_1 \betared e_2$ と表記します。
$\beta$ 簡約を図 \ref{fig:beta-reduction} のように定義します。

\begin{figure}[htbp]
  \[
    \infere{R-Beta}{
      (\lambda x . e_1) \, e_2 \betared [e_2/x]e_1
    }{}
  \]
  \[
    \infere{R-App1}{
      e_1 \, e_2 \betared e_1' \, e_2
    }{
      e_1 \betared e_1'
    }
  \]
  \[
    \infere{R-App2}{
      e_1 \, e_2 \betared e_1 \, e_2'
    }{
      e_2 \betared e_2'
    }
  \]
  \[
    \infere{R-Abs}{
      \lambda x . e \betared \lambda x . e'
    }{
      e \betared e'
    }
  \]
  \caption{$\beta$ 簡約}
  \label{fig:beta-reduction}
\end{figure}

この分数のような表記は、上が仮定で下が結論であるような規則を表しています。
つまり、上に書いてある事が成り立てば、下に書いてある事を成り立つとして良いという事になります。

規則の右側に書いてある \textsc{R-Beta}, \textsc{R-App1}, \textsc{R-App2}, \textsc{R-Abs}
は規則の名前を表しています。

また、\textsc{R-Beta} 規則のように上が空である場合は、
無条件で下に書いてある事を認めて良いという事を表しています。

$\lambda$ 項の評価は $\beta$ 簡約の並びによって表現できます。

\begin{exercise}

以下の $\beta$ 簡約ができる事を示せ。

\[
  (\lambda x . x) \, f \, y \betared f \, y
\]

\subparagraph{解答}

これは、$(\lambda x . x) \, f$ の部分を簡約すれば良さそうです。$[f/x] x = f$ なので、
\[
  \infere{R-Beta}{
    (\lambda x . x) \, f \betared f
  }{}
\]
とできます。この部分式は全体から見ると関数適用の左辺なので、
\[
  \infere{R-App1}{
    (\lambda x . x) \, f \, y \betared f \, y
  }{
    (\lambda x . x) \, f \betared f
  }
\]
とできます。
このようにして、$(\lambda x . x) \, f \, y \betared f \, y$ が正しい $\beta$ 簡約である事が示せます。

また、実際はこの2つの証明の段階を繋げて、
\[
  \infere{R-App1}{
    (\lambda x . x) \, f \, y \betared f \,
 y
  }{
    \infere{R-Beta}{
      (\lambda x . x) \, f \betared f
    }{}
  }
\]
のように表記します。
これによって、複雑な構造を持つ証明を分かりやすく記述する事ができます。

\end{exercise}

\begin{exercise}

以下の $\lambda$ 項を簡約せよ。

\[
  (\lambda x . x \, x) \, (\lambda x . x \, x)
\]

\subparagraph{解答}

外側の関数適用が簡約できます。
$[(\lambda x . x \, x)/x](x \, x) = (\lambda x . x \, x) \, (\lambda x . x \, x)$ なので、
\[
  \infere{R-Beta}{
    (\lambda x . x \, x) \, (\lambda x . x \, x) \betared
	(\lambda x . x \, x) \, (\lambda x . x \, x)
  }{}
\]
となります。

しかし、簡約した結果として自分自身と同じ項が作れてしまいました。
この事から、この項は無限長の $\beta$ 簡約列を作る事ができ、評価が停止しないと分かります。

この例以外にも、無限長の $\beta$ 簡約列を作れる $\lambda$ 項は無数に存在しています。

\end{exercise}

\section{型付き $\lambda$ 計算}

前の練習問題のように、無限長の $\beta$ 簡約列を作る事ができるような $\lambda$ 項を作る事ができます。

また、純粋な $\lambda$ 計算の範囲の話ではありませんが、定数値や定数値に対する演算を含む計算体系では、
特定の種類の定数値のみを対象とできる演算を考えたい場合があります。例えば、
\[
  \text{true} + 1
\]
は加算が数値に対する二項演算であって欲しいとすると、
真偽値 true と整数値 $1$ に対して適用されているため意図に反した計算であると言えます。

このような、停止しない計算や意図に反する計算を「正しくない計算」と見做し、正しくない計算を
表している項を排除する仕組みが\textbf{型付き $\lambda$ 計算 (typed lambda calculus)}です。
\footnote{
ただし、型付き $\lambda$ 計算から派生したプログラミング言語のほとんどは
プログラムが停止しないことや失敗することを許容しています。
しかし、その多くは型によってプログラムの間違いのほとんどを検出し、排除することに成功しています。}

型付き $\lambda$ 計算における\textbf{型 (type)} とは、とある項に対して適用可能な操作の表現です。

\subsection{単純型付き $\lambda$ 計算}

\textbf{単純型付き $\lambda$ 計算 (simply typed lambda calculus, $\lambda^\to$)}は、
型付きラムダ計算の体系の1つです。

単純型付き $\lambda$ 計算の型は、図\ref{fig:stlc-type}のように帰納的に定義されます。

\begin{figure}[htbp]
  \begin{align*}
    t & \bnfcce  \alpha  && \text{(型変数)} \\
      & \bnfvert t \to t && \text{(関数の型)}
  \end{align*}
  \caption{単純型付き $\lambda$ 計算の型}
  \label{fig:stlc-type}
\end{figure}

このように、型変数と関数の型だけからなるものを単純型付き $\lambda$ 計算の型とします。
型は同じ型変数を使うことで同じ型であることを表し、
$\alpha \to \beta$ は型 $\alpha$ の値から型 $\beta$ の値への関数を表します。

例えば、$\lambda x. x$ (恒等関数)は $\alpha \to \alpha$ という型を持ちます。

これまでと同様、型の構造を表すために適宜括弧を使う事があります。
ただし、関数の型は右結合であり、
$t_1 \to (t_2 \to t_3)$ は $t_1 \to t_2 \to t_3$ のように省略できます。

\subsection{型判定}

単純型付き $\lambda$ 計算では、
項に正しく型付けできるという事を\textbf{型判定 (type judgement)} で表します。

型環境 $\Gamma$ の下で項 $e$ の型が $t$ である、ということを
\[ \Gamma \vdash e : t \]
と表記します。

型環境 $\Gamma$ は、変数名と型のペアの集合($x_1 : t_1, x_2 : t_2, \dots, x_n : t_n$)です。
これは型を決定する上での文脈であり、変数の型はここから引いてくる事になります。

型環境 $\Gamma$ に「変数 $x$ の型が $t$ である」
という情報が含まれている事を $x : t \in \Gamma$ と表記します。
また、空の型環境を $\emptyset$ と表記し、型環境 $\Gamma$ に
「変数 $y$ の型が $t$ である」という事を追加した環境を $\Gamma , y : t$ と表記します。
これらは、図\ref{fig:stlc-type-environment}のように定義されます。

\begin{figure}[htbp]
  \[
    \infere{T-Env1}{
      x : t \in \Gamma , x : t
    }{}
  \]
  \[
    \infere{T-Env2}{
      x : t_1 \in \Gamma , y : t_2
    }{
      x \neq y && x : t_1 \in \Gamma
    }
  \]
  \caption{型環境}
  \label{fig:stlc-type-environment}
\end{figure}

規則 \textsc{T-Env1} は、
型環境 $\Gamma, x : t$ に「変数 $x$ の型が $t$ である」が無条件に含まれる、と読みます。

規則 \textsc{T-Env2} は、
型環境 $\Gamma$ に「変数 $x$ の型が $t_1$ である」が含まれていてかつ $x \neq y$ であれば、
型環境 $\Gamma, y : t_2$ に「変数 $x$ の型が $t_1$ である」が含まれる、と読みます。
これは仮定から見れば無関係な変数を追加するための規則であり、
結論から見れば無関係な変数を除去するための規則となっています。

型判定の事を型環境、項、型の間の関係とみなし、
\textbf{型付け関係 (typing relation)} と呼ぶ事もあります。

単純型付き $\lambda$ 計算の型判定は、図\ref{fig:stlc-type-judgement}のようなルールで表せます。
このルールは、$\lambda$ 計算の項の種類1つに対してルールが1つであるということに注意してください。

\begin{figure}[htbp]
  \[
    \infere{T-Var}{
      \Gamma \vdash x : t
    }{
      x : t \in \Gamma
    }
  \]
  \[
    \infere{T-App}{
      \Gamma \vdash e_1 \, e_2 : t_2
    }{
      \Gamma \vdash e_1 : t_1 \to t_2 &
      \Gamma \vdash e_2 : t_1
    }
  \]
  \[
    \infere{T-Abs}{
      \Gamma \vdash \lambda x . e : t_1 \to t_2
    }
    {
      \Gamma, x : t_1 \vdash e : t_2
    }
  \]
  \caption{単純型付き $\lambda$ 計算の型判定}
  \label{fig:stlc-type-judgement}
\end{figure}

規則 \textsc{T-Var} は、型環境 $\Gamma$ に「変数 $x$ の型が $t$ である」が含まれていれば、
型環境 $\Gamma$ の下で式 $x$ の型は $t$ である、と読みます。

規則 \textsc{T-App} は、
型環境 $\Gamma$ の下で式 $e_1$ の型が $t_1 \to t_2$ かつ $e_2$ の型が $t_1$ であれば、
型環境 $\Gamma$ の下で式 $e_1 ~ e_2$ の型は $t_2$ である、と読みます。

規則 \textsc{T-Abs} は、
型環境 $\Gamma , x : t_1$ の下で式 $e$ の型が $t_2$ であれば、
型環境 $\Gamma$ の下で式 $\lambda x . e$ の型が $t_1 \to t_2$ である、と読みます。

また、定数などを含む $\lambda$ 項に型を付ける場合には、
それらの定数や関連する演算に関しての規則を追加する必要があります。
例えば、ブール値(bool 型)と整数値(int 型)と整数値の加算を加えた定義は、
項にブール値と整数値を加え、型に bool 型と int 型を加えた上で、
図\ref{fig:stlc-type-judgement-constants}の型判定規則を追加します。

\begin{figure}[htbp]
  \[
    \infere{T-Bool}{
      \Gamma \vdash c : \text{bool}
    }{
      c \text{はブール値}
    }
  \]
  \[
    \infere{T-Int}{
      \Gamma \vdash c : \text{int}
    }{
      c \text{は整数値}
    }
  \]
  \[
    \infere{T-Plus}{
      \Gamma \vdash e_1 + e_2 : \text{int}
    }{
      \Gamma \vdash e_1 : \text{int} &
      \Gamma \vdash e_2 : \text{int}
    }
  \]
  \caption{定数に関する追加の型判定規則}
  \label{fig:stlc-type-judgement-constants}
\end{figure}

これらの規則は簡単なので、練習も兼ねて自分で読んでみてください。

\subsection{型を付ける}

ここまでに出てきた規則を利用して、実際に $\lambda$ 項に型を付けてみましょう。

\subsubsection{例1 : $\lambda x . x + 1$}

例として、項 $\lambda x . x + 1$ に型を付けてみます。
項の外側から、即ち型判定木の下の方から構築していきましょう。一番外側は $\lambda$ 抽象なので、
\[
  \infere{T-Abs}{
    \emptyset \vdash \lambda x . x + 1 : ?_0 \to ?_1
  }{
    \emptyset , x : ?_0 \vdash x + 1 : ?_1
  }
\]
と書けます。

ただし、ここでの記号 $?$ はまだ確定していない型を表現しています。
同じ番号を振る事で同じ型である事を表現しています。

次に出てきたのは加算なので、
\[
  \infere{T-Abs}{
    \emptyset \vdash \lambda x . x + 1 : ?_0 \to \text{int}
  }{
    \infere{T-Plus}{
      \emptyset , x : ?_0 \vdash x + 1 : \text{int}
    }{
      \emptyset , x : ?_0 \vdash x : \text{int} &
      \emptyset , x : ?_0 \vdash 1 : \text{int}
    }
  }
\]
と書けます。規則 \textsc{T-Plus} によって $?_1$ が int 型に置き換えられた事に注意してください。

次に、左側の項 $x$ の型判定を埋めましょう。規則 \textsc{T-Var} と規則 \textsc{T-Env1} を使うと、
\[
  \infere{T-Abs}{
    \emptyset \vdash \lambda x . x + 1 : \text{int} \to \text{int}
  }{
    \infere{T-Plus}{
      \emptyset , x : \text{int} \vdash x + 1 : \text{int}
    }{
      \infere{T-Var}{
        \emptyset , x : \text{int} \vdash x : \text{int}
      }{
        \infere{T-Env1}{
          x : \text{int} \in \emptyset , x : \text{int}
   		}{}
      } &
      \emptyset , x : \text{int} \vdash 1 : \text{int}
    }
  }
\]
と書けます。ここでも、規則 \textsc{T-Env1} の制約により $?_0$ が int 型に置き換わりました。

最後に、定数1の型判定を埋めます。規則 \textsc{T-Int} を使って、
\[
  \infere{T-Abs}{
    \emptyset \vdash \lambda x . x + 1 : \text{int} \to \text{int}
  }{
    \infere{T-Plus}{
      \emptyset , x : \text{int} \vdash x + 1 : \text{int}
    }{
      \infere{T-Var}{
        \emptyset , x : \text{int} \vdash x : \text{int}
      }{
        \infere{T-Env1}{
          x : \text{int} \in \emptyset , x : \text{int}
   		}{}
      } &
      \infere{T-Int}{
        \emptyset , x : \text{int} \vdash 1 : \text{int}
      }{
        1 \text{は整数値}
	  }
    }
  }
\]
と書けます。これで型判定全体が作れました。

このようにして、
$\lambda x . x + 1$ の型が $\text{int} \to \text{int}$ である事の理由付けができます。

\subsubsection{例2 : $\lambda x . \lambda y . x$}

もう1つの例として、項 $\lambda x . \lambda y . x$ に型を付けてみます。
まず、項全体は $\lambda$ 抽象なので、
\[
  \infere{T-Abs}{
    \emptyset \vdash \lambda x . \lambda y . x : ?_0 \to ?_1
  }{
    \emptyset , x : ?_0 \vdash \lambda y . x : ?_1
  }
\]
と書けます。

次に現れているのも $\lambda$ 抽象なので、
\[
  \infere{T-Abs}{
    \emptyset \vdash \lambda x . \lambda y . x : ?_0 \to ?_2 \to ?_3
  }{
    \infere{T-Abs}{
      \emptyset , x : ?_0 \vdash \lambda y . x : ?_2 \to ?_3
	}{
	  \emptyset , x : ?_0 , y : ?_2 \vdash x : ?_3
	}
  }
\]
と書けます。規則 \textsc{T-Abs} の制約により、$?_1$ が $?_2 \to ?_3$ に置き換わりました。

次に現れているのは変数なので、
\[
  \infere{T-Abs}{
    \emptyset \vdash \lambda x . \lambda y . x : ?_0 \to ?_2 \to ?_0
  }{
    \infere{T-Abs}{
      \emptyset , x : ?_0 \vdash \lambda y . x : ?_2 \to ?_0
	}{
	  \infere{T-Var}{
  	    \emptyset , x : ?_0 , y : ?_2 \vdash x : ?_0
      }{
	    \infere{T-Env2}{
          x : ?_0 \in \emptyset , x : ?_0 , y : ?_2
 		}{
		  x \neq y &
          \infere{T-Env1}{
		    x : ?_0 \in \emptyset , x : ?_0
		  }{}
		}
      }
	}
  }
\]
と書けます。規則 \textsc{T-Env} の制約により $?_0$ と $?_3$ が同一であると分かったため、
$?_3$ を $?_0$ で置き換えました。ただし、これは $?_0$ を $?_3$ で置き換えても構いません。

これで一応型は付けられたように見えますが、記号 $?$ がそのまま残ってしまいました。
これらは、とりあえず型変数で表現する事にします。
今の例であれば、$\alpha \to \beta \to \alpha$ のように書きます。

この推論後に残ってしまう $?$ については、また3章で別の形で出てきます。

\begin{exercise}

以下の項が単純型付き $\lambda$ 計算において型付け不可能である事を示せ。

\begin{enumerate}
  \item $\text{true} + 1$
  \item $(\lambda x . x \, x) \, (\lambda x . x \, x)$
\end{enumerate}

\subparagraph{解答}

まず1番目の項ですが、これは節の最初に出てきた意図に反する項の例です。
このような項は型システムによって排除されるはずです。本当に型付け不可能なのか確かめてみましょう。

まず、項全体は加算になっているので、
\[
  \infere{T-Plus}{
    \emptyset \vdash \text{true} + 1 : \text{int}
  }{
    \emptyset \vdash \text{true} : \text{int} &
    \emptyset \vdash 1 : \text{int}
  }
\]
と書けます。

しかし、ここで $\emptyset \vdash \text{true} : \text{int}$ の部分に注目すると、
このような型判定を作れる規則は存在しません。

次に2番目の項ですが、
これは節の最初に出てきた無限長の $\beta$ 簡約列を作る事ができるような $\lambda$ 項の例です。
本当に型付け不可能なのか確かめてみましょう。

まず、項全体が関数適用になっているので、
\[
  \infere{T-App}{
    \emptyset \vdash (\lambda x . x \, x) \, (\lambda x . x \, x) : ?_1
  }{
    \emptyset \vdash \lambda x . x \, x : ?_0 \to ?_1 &
    \emptyset \vdash \lambda x . x \, x : ?_0
  }
\]
と書けます。

次に、左側の $\lambda$ 抽象に注目すると、
\[
  \infere{T-App}{
    \emptyset \vdash (\lambda x . x \, x) \, (\lambda x . x \, x) : ?_1
  }{
    \infere{T-Abs}{
      \emptyset \vdash \lambda x . x \, x : ?_0 \to ?_1
	}{
      \{x : ?_0\} \vdash x \, x : ?_1
	} &
    \emptyset \vdash \lambda x . x \, x : ?_0
  }
\]
と書けます。ここで、$\{x : ?_0\}$ というのは $\emptyset, x : ?_0$ の簡略表記です。
これ以降も、$\emptyset, x_1 : t_1, \dots, x_n : t_n$ を $\{x_1 : t_1, \dots, x_n : t_n\}$
と表記する事があります。

次に、$x \, x$ の関数適用に注目すると、
\[
  \infere{T-App}{
    \emptyset \vdash (\lambda x . x \, x) \, (\lambda x . x \, x) : ?_1
  }{
    \infere{T-Abs}{
      \emptyset \vdash \lambda x . x \, x : ?_0 \to ?_1
	}{
      \infere{T-App}{
	    \{x : ?_0\} \vdash x \, x : ?_1
      }{
	    \{x : ?_0\} \vdash x : ?_2 \to ?_1 &
		\{x : ?_0\} \vdash x : ?_2
	  }
	} &
    \emptyset \vdash \lambda x . x \, x : ?_0
  }
\]
と書けます。

次に、2つの変数 $x$ に注目すると、
\[
  \infere{T-App}{
    \emptyset \vdash (\lambda x . x \, x) \, (\lambda x . x \, x) : ?_1
  }{
    \infere{T-Abs}{
      \emptyset \vdash \lambda x . x \, x : ?_0 \to ?_1
	}{
      \infere{T-App}{
	    \{x : ?_0\} \vdash x \, x : ?_1
      }{
  	    \infere{T-Var}{
		  \{x : ?_0\} \vdash x : ?_0 \to ?_1
 		}{
		  x : ?_0 \to ?_1 \in \{x : ?_0\}
		} &
 		\infere{T-Var}{
		  \{x : ?_0\} \vdash x : ?_0
		}{
		  \infere{T-Env1}{
            x : ?_0 \in \{x : ?_0\}
		  }{}
		}
	  }
	} &
    \emptyset \vdash \lambda x . x \, x : ?_0
  }
\]
と書けます。規則 \textsc{T-Var} 内の制約により $?_2$ が $?_0$ に置き換わりました。

しかし、ここで $x : ?_0 \to ?_1 \in \{x : ?_0\}$ の部分がまだ書かれていません。
この形を見る限りでは規則 \textsc{T-Env1} が適用されるように見えますが、
もしそうだとすると $?_0 = ?_0 \to ?_1$ であるような型 $?_0$ が無くてはいけません。
しかしそれは $((\dots \to ?_1) \to ?_1) \to ?_1$ のような無限長の型になってしまいます。
\footnote{型推論において、このような無限長の型が作られない事を保証し、
型推論が停止しない事を防ぐための仕組みを出現検査(occurs check)と言います。
これは2章で説明します。}

型の定義は帰納的であるため、このような無限長の型を作る事はできません。
この事から、項 $(\lambda x . x \, x) \, (\lambda x . x \, x)$ が型付けできない事が分かります。

\subparagraph{別解}

帰納法により、「項が型付けできればその項の任意の部分は型付け可能」、
また、その対偶「とある項の部分が型付け不可能であれば全体も型付け不可能」が証明できます。

$\lambda x . x \, x$ 自体が型付け不可能なので、この2つの事実から
$(\lambda x . x \, x) \, (\lambda x . x \, x)$ が型付け不可能であると言えます。

\end{exercise}

\section{型検査}



