
この章では、単純型付き $\lambda$ 計算に一階の多相型を追加した体系の型推論について論じます。

\section{多相の必要性}

単純型付き $\lambda$ 計算に整数値と整数型とタプル(組)の型を追加した体系では、
$(\lambda x. x) \, 12, (\lambda x . x) \, (\lambda x . x)$
という項は $\mathrm{int} \times (\alpha \to \alpha)$ という型で型付けできます。
\footnote{$e_1, e_2$ は $e_1$ と $e_2$ の組の項を表します。
$t_1 \times t_2$ は $t_1$ 型の値と $t_2$ 型の値の組の型を表します。}

しかし、この項の中の恒等関数を括り出して $(\lambda f .(f \, 12, f \, f)) \, (\lambda x . x)$
のように書き換えてやると、型付けできなくなってしまいます。
これは、恒等関数の型が元々
$\mathrm{int} \to \mathrm{int}$ と $(\alpha \to \alpha) \to \alpha \to \alpha$ と
$\alpha \to \alpha$ のように全て一致しないことによる問題です。

これらの型はどれも $t_1 \to t_1$ の形を持っていますが、
それぞれ $t_1$ に入る型が違ってしまうために型が合いません。
これでは任意の型に関する関数などが書けず、そのために同じ定義を何度も書かなければなりません。

この章で扱う多相は、この「任意の型」を扱うための多相です。
この多相性を追加した型付き $\lambda$ 計算について考えてみましょう。

\section{let 項}

前節で説明したような多相性を導入することを明示するための新しい項の要素を、
$\lambda$ 項の定義に追加します。

この新しい項の要素を \textbf{let 項 (let term)} と呼び、
$\letterm{x}{e_1}{e_2}$のように表記します。

この項の簡約の上での意味は $(\lambda x . e_2) \, e_1$ と同一であり、
$e_1$ が多相的に変数 $x$ に束縛されるものとします。

let 項を追加した項を、図\ref{fig:poly-lambda-term}のように定義します。

\begin{figure}[htbp]
  \begin{align*}
    e & \bnfcce  x                   && \text{(変数)} \\
      & \bnfvert (e \, e)            && \text{(関数適用)} \\
      & \bnfvert (\lambda x . e)     && \text{($\lambda$ 抽象)} \\
      & \bnfvert (\letterm{x}{e}{e}) && \text{(let 項)}
  \end{align*}
  \caption{$\lambda$ 計算の項}
  \label{fig:poly-lambda-term}
\end{figure}

追加の省略に関する規則は以下の通りです。

\begin{enumerate}
  \item 項 $e$ の部分項 $\letterm{x}{e_1}{e_2}$ は、$e_2$ の左右に対応する括弧があれば、
        その括弧を省略できます。
\end{enumerate}

\section{置換による多相}



\section{パラメトリック多相}



\section{まとめ}



