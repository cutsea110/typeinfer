
この章では、単純型付き $\lambda$ 計算に多相性を追加した体系の型推論について論じます。

\section{多相の必要性}

単純型付き $\lambda$ 計算に整数値と整数型とタプル(組)の型を追加した体系では、
$(\lambda x. x) \, 12, (\lambda x . x) \, (\lambda x . x)$
という項は $\mathrm{int} \times (\alpha \to \alpha)$ という型で型付けできます。
\footnote{$e_1, e_2$ は $e_1$ と $e_2$ の組の項を表します。
$t_1 \times t_2$ は $t_1$ 型の値と $t_2$ 型の値の組の型を表します。}

しかし、この項の中の恒等関数を括り出して $(\lambda f .(f \, 12, f \, f)) \, (\lambda x . x)$
のように書き換えてやると、型付けできなくなってしまいます。
これは、恒等関数の型が元々
$\mathrm{int} \to \mathrm{int}$ と $(\alpha \to \alpha) \to \alpha \to \alpha$ と
$\alpha \to \alpha$ のように全て一致しないことによる問題です。

これらの型はどれも $t_1 \to t_1$ の形を持っていますが、
それぞれ $t_1$ に入る型が違ってしまうために型が合いません。
これでは任意の型に関する関数などが書けず、そのために同じ定義を何度も書かなければなりません。

この章で扱う多相は、この「任意の型」を扱うための多相です。
この多相性を追加した型付き $\lambda$ 計算について考えてみましょう。

\section{let 項}

前節で説明したような多相性を導入することを明示するための新しい項の要素を、
$\lambda$ 項の定義に追加します。

この新しい項の要素を\textbf{let 項 (let term)} と呼び、
$(\letterm{x}{e_1}{e_2})$ のように表記します。

let 項を追加した項を、図\ref{fig:poly-lambda-term}のように定義します。

\begin{figure}[htbp]
  \begin{align*}
    e & \bnfcce  x                   && \text{(変数)} \\
      & \bnfvert (e \, e)            && \text{(関数適用)} \\
      & \bnfvert (\lambda x . e)     && \text{($\lambda$ 抽象)} \\
      & \bnfvert (\letterm{x}{e}{e}) && \text{(let 項)}
  \end{align*}
  \caption{$\lambda$ 計算の項に let を追加した項}
  \label{fig:poly-lambda-term}
\end{figure}

let 項の簡約の上での意味は $(\lambda x . e_2) \, e_1$ と同一であり、
変数 $x$ が項 $e_1$ に多相的に束縛されるものとします。

$\lambda$ 抽象で束縛されている変数を単相的な変数、let 項で束縛されている変数を多相的な変数と呼びます。

省略に関する追加の規則は以下の通りです。

\begin{enumerate}
  \item 項 $e$ の部分項 $(\letterm{x}{e_1}{e_2})$ は、$e_2$ の左右に対応する括弧があれば、
        その括弧を省略できます。
\end{enumerate}

\begin{note}
ここまでの内容で、省略表記であることが明らかでない let 項の表記は省略されていてはならない。
\end{note}

\section{置換による多相}

\subsection{アルゴリズム}

この節では、多相性を含む新しい項の定義に対する最も簡単な型付けの体系と、
その体系に合った適切な型推論アルゴリズムを定義します。

多相を導入する項、即ち let 項を全て簡約してしまえば、
多相性がどこにも現れない値としての意味\footnote{β簡約を含めた同値関係の意味。}
で等価な項が得られるはずです。

let を含む任意の項から、その項中の let 項を全て簡約した項を計算するアルゴリズムが存在します。
このアルゴリズムの名前を $\mathcal R_\mathrm{let}$ とします。

アルゴリズム $\mathcal R_\mathrm{let}$ は図\ref{fig:algorithm-rlet}のように定義できます。

\begin{figure}[htbp]
  \begin{align*}
    \mathcal R_\mathrm{let}(x) &= x \\
    \mathcal R_\mathrm{let}(e_1 e_2) &=
      \mathcal R_\mathrm{let}(e_1) \, \mathcal R_\mathrm{let}(e_2) \\
    \mathcal R_\mathrm{let}(\lambda x . e) &=
      \lambda x . \mathcal R_\mathrm{let}(e) \\
    \mathcal R_\mathrm{let}(\letterm{x}{e_1}{e_2}) &=
      \mathcal R_\mathrm{let}(e_2 [x := \mathcal R_\mathrm{let}(e_1)])
  \end{align*}
  \caption{アルゴリズム $\mathcal R_\mathrm{let}$}
  \label{fig:algorithm-rlet}
\end{figure}

このアルゴリズムは再帰に関して項の構造もしくは項中の let 項の数が減少しているため、
正しい帰納的定義になっています。

これによって let を含む項を let を含まない等価な項に変換し、
2章で示したアルゴリズムによって型推論をすることで、let を含む項の型を推論できます。

ただし、簡約をしてしまうと変数の捕獲が問題になってしまう場合があります。
適切な $\alpha$ 変換をするアルゴリズムを書いても良いのですが、
制約を生成するアルゴリズムに少し変更を加えることによって、
より実装に向いた型推論アルゴリズムを得られます。

let を含む項の場合では、導入される変数が二種類あります。
一方は単相的に束縛され、もう一方は多相的に束縛されます。

型環境の上でこれらを区別するようにしましょう。
単相の変数は型だけを持っておけば良いはずです。一方、多相の変数は型環境と項を持っておき、
変数が使われる度に制約を生成することで適切な制約集合が得られます。
これらを区別するため、前者は持つ型を $t$ としたときに $\mathrm{Mono}(t)$ と、
後者は型環境と項をそれぞれ $\Gamma$ と $e$ としたときに $\mathrm{Poly}(\Gamma, e)$ と表記します。
\footnote{それぞれ、monomorphic (単相的)と polymorphic (多相的)の略。}

let を含む項に対する制約生成アルゴリズム $\mathcal{C}_P$ を、
図\ref{fig:algorithm-cp}のように定義します。

\begin{figure}[htbp]
  \begin{align*}
    \mathcal C_P(\Gamma, x) &=
      \begin{cases}
        \mathit{failure}          & (x \notin \mathit{dom}(\Gamma)) \\
        t                         & (\mathrm{Mono}(t) = \Gamma(x)) \\
        \mathcal C_P(\Gamma', e) & (\mathrm{Poly}(\Gamma', e) = \Gamma(x)) \\
      \end{cases} \\
    \mathcal C_P(\Gamma, e_1 \, e_2) &=
      \begin{array}[t]{l}
        \mathrm{let}
          \begin{array}[t]{l}
            (c_1, t_1) = \mathcal C_P(\Gamma, e_1) \\
            (c_2, t_2) = \mathcal C_P(\Gamma, e_2) \\
            t_3 = \mathrm{fresh}
          \end{array} \\
        \mathrm{in} (c_1 \cup c_2 \cup \{t_1 = t_2 \to t_3\}, t_3)
      \end{array} \\
    \mathcal C_P(\Gamma, \lambda x . e) &=
      \begin{array}[t]{l}
        \mathrm{let}
          \begin{array}[t]{l}
            t_1 = \mathrm{fresh} \\
            (c, t_2) = \mathcal C_P((\Gamma, x : \mathrm{Mono}(t_1)), e)
          \end{array} \\
        \mathrm{in} (c, t_1 \to t_2)
      \end{array} \\
    \mathcal C_P(\Gamma, \letterm{x}{e_1}{e_2}) &=
      \mathcal C_P((\Gamma, x : \mathrm{Poly}(\Gamma, e_1)), e_2)
  \end{align*}
  \caption{方程式の生成アルゴリズム $\mathcal C_P$}
  \label{fig:algorithm-cp}
\end{figure}

アルゴリズム $\mathcal C$ からの主な変更点は三箇所です。

\begin{itemize}
  \item 項が変数であった場合に、それが単相の変数か多相の変数かによって計算が変わります。

        単相の変数であれば、型環境から得られる型をそのまま返します。
        多相の変数であれば、型環境から得られる型環境と項を用いて制約集合を生成します。

  \item $\lambda$ 抽象に関する制約の生成で、
        型環境に追加される情報が $\mathrm{Mono}(t)$ の形になる。

  \item let 項に対する制約の生成が追加されました。

        型環境 $\Gamma$ と $(\letterm{x}{e_1}{e_2})$ という項を取る場合、
        $\Gamma$ に変数 $x$ と $\mathrm{Poly}(\Gamma, e_1)$ の対応付けを追加した型環境を作り、
        その型環境と $e_2$ を使って制約の生成を行います。
\end{itemize}

これと第2章で示した単一化アルゴリズムを用いて、
let 項を含む項に対する適切な型推論アルゴリズムを作れます。

\subsection{実装}

多相性を実現するための制約生成アルゴリズムを実装してみましょう。

型環境をリスト\ref{list:ocaml-def-poly-type-environment}のように定義します。

\begin{lstlisting}[caption=型環境の定義, label=list:ocaml-def-poly-type-environment]
type assump_elem
    = Monovar of ty
    | Polyvar of (assump_elem String.Map.t * term)
type assump = assump_elem String.Map.t
\end{lstlisting}

\texttt{assump\_elem} 型は型環境の一つの要素を表しています。

\texttt{Monovar} は単相変数用のコンストラクタ、\texttt{Polyvar} は多相変数用のコンストラクタです。

\texttt{assump} 型は型環境を表しています。
第2章での定義と違い、\texttt{assump\_elem} 型を用いていることに注意してください。

\texttt{constraints} 関数をリスト\ref{list:ocaml-poly-constraints}のように定義します。

\begin{lstlisting}[caption=制約の生成, label=list:ocaml-poly-constraints]
let rec constraints (n : int) (env : assump) :
    term -> (int * tconst list * ty) option =
  function
    | EVar str ->
      begin match String.Map.find env str with
        | Some (Monovar t) -> Some (n, [], t)
        | Some (Polyvar (env', term)) -> constraints n env' term
        | None -> None
      end
    | EApp (term1, term2) ->
      begin match constraints (succ n) env term1 with
        | Some (n1, c1, t1) ->
          begin match constraints n1 env term2 with
            | Some (n2, c2, t2) ->
              let tn = TVar n in
              Some (n2, (t1, TFun (t2, tn)) :: c1 @ c2, tn)
            | None -> None
          end
        | None -> None
      end
    | EAbs (ident, term) ->
      begin
        let tn = TVar n in
        let newenv = String.Map.add ident (Monovar tn) env in
        match constraints (succ n) newenv term with
          | Some (n', c, t) -> Some (n', c, TFun (tn, t))
          | None -> None
      end
    | ELet (ident, term1, term2) ->
      let newenv = String.Map.add ident (Polyvar (env, term1)) env in
      constraints n newenv term2
\end{lstlisting}

これも、2章で示した \texttt{constraints} 関数からの変更点はアルゴリズムと変更と同様の三箇所です。

\section{パラメトリック多相}

置き換えによる多相性を含む $\lambda$ 計算の型推論は、この本で目標とする範囲の推論アルゴリズムと
完全に一致していると言えます。

しかし、この方法では時間的なコストが大きくなってしまう、
型が一致しないときに問題のある項を適切に示せないなどの問題があり、
多くの言語処理系では別の方法を採用しています。

ここでは、そのより良い方法である\textbf{パラメトリック多相 (parametric polymorphism)} を用いた
型付けの体系と、それに基いた推論規則について説明します。

パラメトリック多相の話に入るために、多相性の関わる項の例を幾つか考えてみます。

まずは、恒等関数 $\lambda x . x$ について考えましょう。
この項の最も広い意味での型は、$\alpha \to \alpha$ です。

$\letterm{i}{\lambda x . x}{\dots}$ のように書くことで、
$\dots$ の中で恒等関数 $i$ を多相的に扱うことができます。
このとき、その全ての $i$ の型はどれも $t_1 \to t_1$ の形をしているはずです。
つまり、多相的な恒等関数は、任意の型 $\alpha$ に関して $\alpha \to \alpha$ という型を持つと言えます。
パラメトリック多相では、
このような型を記号 $\forall$ を用いて $\forall \alpha . \alpha \to \alpha$ のように表記します。

この例のように、項 $e$ が型 $t$ で型付けできるとき、
項 $e$ に多相的に束縛された変数 $x$ は幾つかの任意の型に関して型 $t$ を持つと言えます。
項 $x$ の型は、変数 $x$ の型の任意の型全てに適切に型を代入することによって得られます。

さて、ここでどの型変数を任意の型として良いのかという問題がありますが、
全ての型変数を任意の型として良いわけではありません。

もう一つ、
新しい例として $\lambda a . \letterm{f}{\lambda b . (a, b)}{(f, f)}$ という項を考えてみましょう。

ここで、$(e_1, e_2)$ という形で表わされている部分は $e_1$ と $e_2$ の直積(タプル)です。
型 $t_1$ と型 $t_2$ の直積型を $(t_1 \times t_2)$ という形で表現します。
直積に関する追加の型付け規則 \textsc{T-Pro} を、図\ref{fig:product-type-judgement}のように定義します。

\begin{figure}[htbp]
  \[
    \infere{T-Pro}{
      \Gamma \vdash (e_1, e_2) : (t_1 \times t_2)
    }{
      \Gamma \vdash e_1 : t_1 &
      \Gamma \vdash e_2 : t_2
    }
  \]
  \caption{直積に関する型判定}
  \label{fig:product-type-judgement}
\end{figure}

さて、今考えている項は、
$\alpha \to ((\gamma \to (\alpha \times \gamma)) \times (\delta \to (\alpha \times \delta)))$
で型付けでき、かつこれが最も一般的な型です。
これは二つの $f$ の直積を結果としていますが、それぞれの $f$ に相当する型は
$\gamma \to (\alpha \times \gamma)$ と $\delta \to (\alpha \times \delta)$ になっているため、
fの型は $\forall \beta . \beta \to (\alpha \times \beta)$ となっているはずです。
$\beta$ は任意の型となっていますが、$\alpha$ はなっていません。

これは、$\alpha$ が let の外側の型環境に含まれていることによる制限です。
外にもある型変数を勝手に付け替えてしまえば、型の整合性が取れなくなってしまいます。

よって、多相変数の束縛において $\forall$ の後に書かれる型変数は、その多相変数が束縛されている
項やその部分項の型以外の部分(let の外側の型環境)で出現してはいけないことになります。
この他に制約はありませんが、型中に一度も出現していない型変数を任意の型としてしまっても意味が無いので、
多相変数が束縛されている項の型に含まれ、かつその項やその部分項の型以外の部分で全く出現していない
型変数を任意の型として良いと考えるのが良いでしょう。

\subsection{理論}



\subsection{アルゴリズム}



\subsection{実装}



\section{まとめ}



