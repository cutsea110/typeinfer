\documentclass[b5paper]{jsbook}

\usepackage{amsmath}
\usepackage{amssymb}
\usepackage{proof}

\newlength{\lengthwithlength}
\newcommand{\bnfvert}{\settowidth{\lengthwithlength}{::=}\mathrel{\makebox[\lengthwithlength][c]{$|$}}}
\newcommand{\bnfcce}{\mathrel{::=}}

\title{Typing $\lambda^\to$ (C80 版)}
\author{坂口和彦}

\begin{document}

\maketitle

\pagenumbering{roman}
\tableofcontents
\newpage
\pagenumbering{arabic}

\chapter{序論}

\section{まえがき}

この本は、単純型付きラムダ計算からパラメトリック多相入りの型システム程度の範囲での型推論について、
体系的かつ理論と実装に一対一の対応が現れる形でまとめた本です。

型推論の実装は関数型プログラミング言語 Objective Caml (OCaml) 3.12.0
とその標準ライブラリによって記述されています。
読者は関数型プログラミングや OCaml のエキスパートである必要はありませんが、ある程度 Haskell や OCaml
でのプログラムの書き方を理解している事を前提としています。

\section{この本が書かれた背景}

著者は、この本を書く前にもラムダプラス+という型推論をテーマにした本を書いています。
しかし、ラムダプラス+で出した型推論の例は対象としている言語が大きく、
入門には適さないなどの問題がありました。

その問題を解決しよう。というのがこの本を書く上での目標です。
より明確な目標としては、以下の3つが挙げられます。

\begin{itemize}
  \item 理論が先、実装が後になるように内容を構築する。理由が後付けではいけない。
  \item 段階的に理論を拡張していくように書く。
  \item 実装は理論に忠実に。実装の差分も理論上の拡張に対応するように書く。
\end{itemize}

\chapter{$\lambda$ 計算入門}

$\lambda$ 計算は計算体系の一つで、計算を\textbf{関数}と\textbf{関数適用}で表します。
現在使われている多くの関数型プログラミング言語の基礎には、$\lambda$ 計算があると言えます。

次章から書く型推論の対象となる言語も、
最初は $\lambda$ 計算のシンプルな型システムとして知られる単純型付き $\lambda$ 計算です。

この章では、前提知識として必要な $\lambda$ 計算について説明します。

\section{$\lambda$ 計算}

$\lambda$ 計算では、関数を記号 $\lambda$ を使って表記します。
例えば、値を取ってその値をそのまま返す関数は、 \[ \lambda x. x+1 \] と表記します。

作った関数を値に適用する仕組みが必要です。
関数適用は、関数の式と引数の式を順番に並べて表記します。
例えば、引数をそのまま返す関数にその関数自身を適用するには、
\[ (\lambda x. x) (\lambda y. y) \] と表記します。

このような計算そのものを表現する物を項と呼びます。
$\lambda$ 計算の項は、図\ref{fig:lambda-term}のように帰納的に定義されます。

\begin{figure}[htbp]
  \begin{align*}
    e & \bnfcce  x             && \text{(変数)} \\
      & \bnfvert e ~ e         && \text{(関数適用)} \\
      & \bnfvert \lambda x . e && \text{($\lambda$ 抽象)}
  \end{align*}
  \caption{定義 : $\lambda$ 計算の項}
  \label{fig:lambda-term}
\end{figure}

このように、変数、巻数適用、$\lambda$ 抽象からなる物だけが $\lambda$ 計算の\textbf{項}と呼ばれます。
項の事を\textbf{式}という場合もあります。

項の構造を明確にしつつ表記を簡単にするため、幾つかのルールがあります。



\section{$\beta$ 簡約}



\section{型付き $\lambda$ 計算}



\end{document}
