\documentclass[b5paper]{jsbook}

\usepackage{amsmath}
\usepackage{amssymb}
\usepackage{listings}
\usepackage{theorem}
\usepackage{proof}
\usepackage[dvips]{hyperref}

\newlength{\lengthwithlength}
\newcommand{\bnfvert}{\settowidth{\lengthwithlength}{::=}\mathrel{\makebox[\lengthwithlength][c]{$|$}}}
\newcommand{\bnfcce}{\mathrel{::=}}

\newcommand{\betared}{\to_\beta}
\newcommand{\infere}[3]{\infer[\textsc{[#1]}]{#2}{#3}}

{
\theorembodyfont{\upshape}
\newtheorem{exercise}{練習問題}[chapter]
\newtheorem{answer}{回答}[chapter]
\theoremstyle{break}
\newtheorem{definition}{定義}[chapter]
}

\lstset{
  basicstyle=\small\tt,
  keywordstyle=\bf,
  identifierstyle=,
  commentstyle=,
  stringstyle=,
  numberstyle=,
  emphstyle=,
  language=[Objective]caml,
  frame=trbl
}

\renewcommand{\lstlistingname}{リスト}

\setcounter{tocdepth}{2}

\title{Algorithm $\mathcal W$ 入門}
\author{坂口和彦}

\begin{document}

\maketitle

\pagenumbering{roman}

\section*{前書き}

この本は、単純型付きラムダ計算からパラメトリック多相入りの型システム程度の範囲での型推論について、
体系的かつ理論と実装に一対一の対応が現れる形でまとめた本です。
最後まで読めば Hindley-Milner 型推論アルゴリズム(Algorithm $\mathcal W$)
の現実的な実装を理解できるということを目標としています。

そもそも、Hindley-Milner 型推論アルゴリズムを学ぶことにはどのような意味があるのでしょうか。
著者は、この問いに対する答えは2つあると考えています。

まず1つは、静的型付けの型推論を持つ関数型プログラミング言語処理系を作る上で、
避けては通れない道であるということです。
しかし、それだけの理由で型推論を学ぶ必要がある人はごく僅かしかいません。

もう1つの理由は、型推論の仕組みを知ることで、
型推論の機能を持つ言語処理系の型に関するメッセージが圧倒的に理解しやすくなるという点にあります。

OCaml や Haskell、その他類似の言語を使っていて、
型エラーの意味する所が良く分からないということがあると思います。
そのエラーは、型推論の過程で整合性が取れないと分かった部分を示しているのです。
よって、型推論アルゴリズムを学ぶことで、
型エラーに対する理解がより一層深くなるのではないかと考えられます。
また、Haskell ではトップレベルの束縛に型を書くのが正しい習慣とされていますが、
それも型推論の仕組みを逆手に取り、分かりづらいエラーを抑制する意味があります。

このような理由から、静的型付けの関数型言語を使っているプログラマにとって、
型推論を学ぶことはとても大きな意味を持つと考えられます。

\section*{使用言語とライブラリ}

型推論の実装は関数型プログラミング言語 OCaml 3.12.0、
OCaml の標準ライブラリ、Jane Street's core によって記述されています。

\section*{対象読者について}

読者は関数型プログラミングや OCaml のエキスパートである必要はありませんが、
ある程度 Haskell や OCaml での関数型言語らしいプログラムの書き方を理解していることを前提としています。

\tableofcontents

\newpage

\pagenumbering{arabic}

\chapter{$\lambda$ 計算入門}


本章では、$\lambda$ 計算の基礎を説明します。
$\lambda$ 計算は、関数の計算に関する理論を扱うための計算体系です。
現在使われている多くの関数型プログラミング言語の基礎には、$\lambda$ 計算があると言えます。

次章から書く型推論の対象となる言語も、
最初は $\lambda$ 計算のシンプルな型システムとして知られる単純型付き $\lambda$ 計算です。

この章では、前提知識として必要な $\lambda$ 計算と単純型付き $\lambda$ 計算について説明します。

\section{関数}

\textbf{関数 (function)}は、与えられた値に依存して値が一意に決定されるような対応を表します。
例えば、与えられた数値に $1$ を足した値を対応させる関数 $s$ は、
\[
  s(x) = x+1
\]
のように表記します。

上のように定義された関数 $s$ に値$2$を与えるには、
\[
  s(2)
\]
のように表記します。また、この値は $2+1$、即ち$3$となります。

さて、このような関数を値として扱うことを考えてみましょう。
関数名のみを書くことで、関数の値を指すものとします。
値は関数に与えられるので、関数も関数に与えられるはずです。
例えば、
\[
  a(f, x) = f(x)
\]
とすると、関数 $f$ と値 $x$ に依存し、
関数 $f$ に値 $x$ を与えた値を対応させる関数 $a$ を書けます。

$\lambda$ 計算では、関数を $\lambda$ という記号を使って
$\lambda \text{変数} . \text{変数に依存する式}$ と表記します。
例えば、関数 $s$ は、
\[
  \lambda x . x + 1
\]
と表記します。\footnote{本来の $\lambda$ 計算には数値や加算は含まれませんが、
分かりやすいように適宜このような表記を使います。}。

関数に値を与えることを、\textbf{関数適用 (function application)}
もしくは単に\textbf{適用 (application)} といいます。
$\lambda$ 計算の関数適用は、関数と引数を順番に並べて表記します。
引数部分を括弧で括る必要はありません。
例えば、関数 $(\lambda x . x + 1)$ に値 $2$ を与えるには、
\[
  (\lambda x . x + 1) \, 2
\]
と表記します。左が関数、右が引数を表しています。

関数を値として扱うような関数も問題なく扱えます。例えば、関数適用の関数は、
\[
  \lambda f . \lambda x . f \, x
\]
と書けます。

この関数は元々2引数の関数ですが、関数を返す関数のように書かれています。
これは、2引数の関数を最初の引数を取って残りの1引数関数が求まる関数に変換しています。
3引数の関数 $f(x, y, z)$ も、$\lambda x . \lambda y . \lambda z . f(x, y, z)$
のように変換できます。4引数以上でも、同様の変換を適用できます。

このように、関数を結果とする関数によって表現された多変数関数を
\textbf{カリー化された関数 (curried function)} といいます。
カリー化によって、どのような多変数関数も複数の1引数関数で表現できます。

これらの記号 $\lambda$ で関数を表すような計算の表現を
$\lambda$ 計算の\textbf{項 (term)} もしくは\textbf{$\lambda$ 項 ($\lambda$-term)} と呼びます。

\section{$\lambda$ 計算}

この節では、$\lambda$ 項を定義します。

変数の集合は、可算無限集合であるものとします。
即ち、変数の集合は自然数と一対一の対応を持ちます。
変数を表すメタ変数としては、$x, y, z, \dots$ などを用います。

$\lambda$ 計算の項を、図\ref{fig:lambda-term}のように帰納的に定義します。

\begin{figure}[htbp]
  \begin{align*}
    e & \bnfcce  x               && \text{(変数)} \\
      & \bnfvert (e \, e)        && \text{(関数適用)} \\
      & \bnfvert (\lambda x . e) && \text{($\lambda$ 抽象)}
  \end{align*}
  \caption{$\lambda$ 計算の項}
  \label{fig:lambda-term}
\end{figure}

このような表現の形式を、\textbf{BNF (Backus-Naur form)} といいます。
この表記では、構造の名前、$::=$、中身として取り得る表現の列を $|$ で区切った表記を並べて書きます。
BNF は良く構文の表現に利用されますが、本書では主に構造の表現に利用しています。

図\ref{fig:lambda-term}の定義を分かりやすく書き直すと、以下のような意味になります。

\begin{itemize}
  \item 変数 $x$ は $\lambda$ 項である。
  \item $\lambda$ 項 $e_1$ と $\lambda$ 項 $e_2$ から作られる式 $(e_1 \, e_2)$ は $\lambda$ 項である。
  \item 変数 $x$ と $\lambda$ 項 $e$ から作られる式 $(\lambda x . e)$ は $\lambda$ 項である。
  \item 以上の規則で作れる $\lambda$ 項のみが、正しい $\lambda$ 項である。
\end{itemize}

項の定義に $\lambda$ 抽象が含まれていますが、これは前節までの関数に対応します。

項のことを\textbf{式}という場合もあります。

また、図\ref{fig:lambda-term}の定義を見て分かる通り、
純粋な $\lambda$ 計算には整数値や論理値などの定数が存在しません。
しかし、整数値や論理値などを関数を使ってエンコードできると知られています。
ここでは分かりやすいように適宜整数などを使うことがあります。

$\lambda$ 項 $e$ の一部分となっているような項を、$e$ の
\textbf{$\lambda$ 部分項 ($\lambda$-subterm)} もしくは単に\textbf{部分項 (subterm)} と呼びます。

$\lambda$ 項において、変数 $x$ の出現とは、項の上で変数 $x$ が現れる位置を意味します。
とある変数 $x$ の出現が $\lambda$ 抽象 $(\lambda x . e)$ の部分項であるとき、
変数 $x$ の出現は\textbf{束縛されている (bound)} といい、
そうでない変数 $x$ の出現は\textbf{自由 (free)} であるといいます。

束縛された出現を持つ変数を\textbf{束縛変数 (bound variable)} といい、
自由な出現を持つ変数を \textbf{自由変数 (free variable)} といいます。

$\lambda$ 項の表記を簡単にするための省略の規則を以下に示します。

\begin{enumerate}
  \item 項 $e$ の部分項 $((e_1 \, e_2) \, e_3)$ は $(e_1 \, e_2 \, e_3)$ と省略できます。
        すなわち、関数適用の構文は左結合です。
  \item 項 $e$ の部分項 $(\lambda x . e_1)$ は、$e_1$ の左右に対応する括弧があれば、
        その括弧を省略できます。
  \item 項 $e$ が $(\dots)$ のように左右を対応する括弧で囲まれているとき、その括弧を省略できます。
        これは、一番外側の括弧を省略するための規則なので、
        部分項に対する規則ではないということに注意してください。
\end{enumerate}

一番目の規則を繰り返し適用することで、カリー化された多変数関数の適用の形をしている部分項
$(\dots((f \, x_1) \, x_2) \dots x_n)$ は $(f \, x_1 \, x_2 \dots x_n)$ のように省略できます。

また、通常は以下の追加の規則も使います。

\begin{enumerate}
  \setcounter{enumi}{3}
  \item 項 $e$ の部分項 $(\lambda x_1. (\dots (\lambda x_n . e_1) \dots))$ は
        $(\lambda x_1 \dots x_n . e_1)$ と省略できます。
\end{enumerate}

本書では、この四番目の規則による省略は行いません。

\begin{note}
ここまでの内容で、省略表記であることが明らかでない項の表記は省略されていてはならない。
\end{note}

\begin{exercise}

次のうち、$\lambda$ 計算の項として正しくない表現はどれか。

\begin{enumerate}
  \item $\lambda x . x$
  \item $\lambda f . f \, x$
  \item $f \, (\lambda x)$
\end{enumerate}

\subparagraph{解答}

\begin{enumerate}
  \item $x$ が変数として正しく、$\lambda x . x$ 全体も $\lambda$ 抽象として正しいので
        正しい項になっています。
  \item $f$ と $x$ がそれぞれ変数として正しく、$f \, x$ が関数適用として正しく、
        $\lambda f . f \, x$ 全体も $\lambda$ 抽象として正しいので正しい項になっています。
  \item $f$ と $x$ はそれぞれ変数として正しい形になっていますが、
        $\lambda x$ は正しい $\lambda$ 計算の項の形をしていません。
        よって、$f \, (\lambda x)$ 全体も正しい $\lambda$ 計算の項ではありません。
\end{enumerate}

$\lambda$ 計算の項として正しくない表現は、三番目の $f \, (\lambda x)$ です。

\end{exercise}

\begin{exercise}

次の $\lambda$ 項の下線部の変数は、束縛変数と自由変数のどちらか。

\begin{enumerate}
  \item $\lambda x . \lambda y . \underline{x}$
  \item $\lambda y . (\lambda f . f \, x) \, \underline{f}$
\end{enumerate}

\subparagraph{解答}

\begin{enumerate}
  \item 下線部の変数 $x$ の出現が $\lambda x . \lambda y . x$ の部分項となっているので、束縛変数です。
  \item 下線部の変数 $f$ の出現は $\lambda f . e$ の形の項の部分項となっていないので、自由変数です。
        $\lambda f . f \, x$ という部分項がありますが、
        下線分の変数 $f$ の出現がこの項の部分項でないことに注意してください。
\end{enumerate}

\end{exercise}

\section{$\alpha$ 同値}

\textbf{$\alpha$ 同値 ($\alpha$-equivalent)}は、$\lambda$ 項の構文上の同値関係です。
項 $e_1$ と項 $e_2$ が $\alpha$ 同値であるということを記号 $\alphaeq$ を用いて
$e_1 \alphaeq e_2$ と表記します。

$\alpha$ 同値の基本は、とある項の束縛変数の名前を置き換えても、
それは同じ項であるということです。例えば、$\lambda x . x$ は $\lambda y . y$ と $\alpha$ 同値です。

この関係を定義するためには、置き換える対象の束縛変数を束縛している $\lambda$ 抽象
$\lambda x . e$ に関して、項 $e$ に自由出現する変数 $x$ を全て置き換えた項を考える必要があります。

まず、この置き換えについて考えましょう。項に自由出現する特定の変数を別の項で置き換える操作を
\textbf{代入 (substitution)} と呼びます。

自由変数 $x$ を全て項 $e$ に置き換える代入を
\[
  [x := e]
\]
と表記します。

項 $e_1$ に代入 $[x := e_2]$ を適用した項、即ち項 $e_1$ 中の自由変数 $x$ を全て項 $e_2$ に置き換える
ことによって得られる項を
\[
  e_1 [x := e_2]
\]
と表記します。

代入の適用は、図 \ref{fig:lambda-substitute} のようにも定義できます。

\begin{figure}[htbp]
  \begin{align*}
    y [x := e_1] & = \left \{
      \begin{array}{ll}
        e_1 & (x = y) \\
        y & (x \neq y)
      \end{array}
      \right. \\
    e_2 \, e_3 [x := e_1] & = (e_2 [x := e_1]) \, (e_3 [x := e_1]) \\
    (\lambda y . e_2) [x := e_1] & = \left \{
      \begin{array}{ll}
        \lambda y . (e_2 [x := e_1]) & (x \neq y) \\
        \lambda y . e_2 & (x = y)
      \end{array}
      \right.
  \end{align*}
  \caption{代入}
  \label{fig:lambda-substitute}
\end{figure}

$\alpha$ 同値の関係を、図\ref{fig:alpha-equivalent}のように定義します。

\begin{figure}[htbp]
  \[
    \infere{Eq-Alpha}{
      (\lambda x . e) \alphaeq (\lambda y . e [x := y])
    }{
      \text{項 $e$ に変数 $y$ が自由出現しない}
    }
  \]
  \[
    \infere{Eq-Refl}{
      e \alphaeq e
    }{}
  \]
  \[
    \infere{Eq-Sym}{
      e' \alphaeq e
    }{
      e \alphaeq e'
    }
  \]
  \[
    \infere{Eq-Trans}{
      e_1 \alphaeq e_3
    }{
      e_1 \alphaeq e_2 & e_2 \alphaeq e_3
    }
  \]
  \[
    \infere{Eq-App}{
      e_1 \, e_2 \alphaeq e_1' \, e_2'
    }{
      e_1 \alphaeq e_1' & e_2 \alphaeq e_2'
    }
  \]
  \[
    \infere{Eq-Abs}{
      \lambda x . e \alphaeq \lambda x . e'
    }{
      e \alphaeq e'
    }
  \]
  \caption{$\alpha$ 同値}
  \label{fig:alpha-equivalent}
\end{figure}

この分数のような表記は、上が仮定で下が結論であるような規則を表しています。
つまり、上に書いてあることが成り立てば、下に書いてあることを成り立つとして良いということになります。

規則の右側に書かれている \textsc{Eq-Alpha}, \textsc{Eq-Refl}, \textsc{Eq-Sym}, \textsc{Eq-Trans},
\textsc{Eq-App}, \textsc{Eq-Abs} は、規則の名前を表しています。

規則を一つずつ見てみましょう。

規則 \textsc{Eq-Alpha} は、束縛変数の置き換えの規則です。
中身は、項 $e$ に変数 $y$ が自由出現しないなら、項 $\lambda x . e$ と $\lambda x . e [x := y]$ は
$\alpha$ 同値であるということを表しています。
項 $e$ に変数 $y$ が自由出現してはいけないのは、そのような変数に置き換えてしまうと
自由出現していた変数がその $\lambda$ 抽象で束縛されてしまい、意味が変わってしまうからです。

規則 \textsc{Eq-Refl}, \textsc{Eq-Sym}, \textsc{Eq-Trans} は同値関係としての性質、
それぞれ反射性、対称性、推移性を表しています。

規則 \textsc{Eq-App} は、関数適用の左辺と右辺をそれぞれ $\alpha$ 同値な項に置き換えても、
全体として $\alpha$ 同値であるという規則です。

規則 \textsc{Eq-Abs} は、$\lambda$ 抽象の内側の項を $\alpha$ 同値な項に置き換えても、
全体として $\alpha$ 同値であるという規則です。

本書ではこれ以降 $\alpha$ 同値性を構文的な等しさとして扱い、
$=_\alpha$ も紛らわしさなどの問題がなければ単に $=$ と表記します。

\begin{exercise}

以下の $\alpha$ 同値関係が成り立つことを示せ。

\[
  \lambda f . \lambda x . f \, (f \, x) \alphaeq
  \lambda f . \lambda y . f \, (f \, y)
\]

\subparagraph{解答}

この $\alpha$ 同値関係では、左辺の束縛変数 $x$ を $y$ に置き換えています。
この変数 $x$ を束縛している $\lambda$ 抽象のみに着目すると、
\[
  \lambda x . f \, (f \, x) \alphaeq \lambda y . f \, (f \, y)
\]
という $\alpha$ 同値関係が成立しているはずです。
項 $f \, (f \, x)$ に変数 $y$ は自由出現しておらず、
$f \, (f \, x) [x := y] = f \, (f \, y)$ が成立するので、規則 \textsc{Eq-Alpha} を使うと
\[
  \infere{Eq-Alpha}{
    \lambda x . f \, (f \, x) \alphaeq
    \lambda y . f \, (f \, y)
  }{
    \text{項 $f \, (f \, x)$ に変数 $y$ が自由出現しない}
  }
\]
と書けます。

上で作れた $\alpha$ 同値関係は、
全体から見れば同じ形をした $\lambda$ 抽象の内側の項に関する関係です。
よって、規則 \textsc{Eq-Abs} を使うと
\[
  \infere{Eq-Abs}{
    \lambda f . \lambda x . f \, (f \, x) \alphaeq
    \lambda f . \lambda y . f \, (f \, y)
  }{
    \lambda x . f \, (f \, x) \alphaeq
    \lambda y . f \, (f \, y)
  }
\]
と書けます。

このようにして、
$\lambda f . \lambda x . f \, (f \, x) \alphaeq \lambda f . \lambda y . f \, (f \, y)$
は正しい $\alpha$ 同値関係であると示せます。

実際はこの二つの証明の段階を繋げて、証明全体を
\[
  \infere{Eq-Abs}{
    \lambda f . \lambda x . f \, (f \, x) \alphaeq
    \lambda f . \lambda y . f \, (f \, y)
  }{
    \infere{Eq-Alpha}{
      \lambda x . f \, (f \, x) \alphaeq
      \lambda y . f \, (f \, y)
    }{
      \text{項 $f \, (f \, x)$ に変数 $y$ が自由出現しない}
    }
  }
\]
のように表記します。
この表記によって、複雑な構造を持つ証明を分かりやすく記述できます。

\end{exercise}

\section{変数の捕獲}

以下のような代入の適用を考えてみましょう。

\[
  (\lambda f . f \, x) [x := f \, y]
\]

これは、
\[
  (\lambda f . f \, (\underline{f} \, y))
\]
という項になります。しかし、下線部の $f$ が代入された項の一部であるにも関わらず、
束縛変数になってしまいました。このとき、変数 $f$ は\textbf{捕獲 (capture)} されたといいます。

一方、$\lambda f . f \, x$ と $\alpha$ 同値な項 $\lambda g . g \, x$ であれば、
\[
  (\lambda g . g \, (\underline{f} \, y))
\]
のようになるため、変数 $f$ は捕獲されません。

変数の捕獲によって、本来の意味と異なる項ができてしまうという問題が考えられます。
この問題を防ぐため、$\lambda$ 計算で扱う項は、$\alpha$ 同値である項を選んでくることによって、
束縛変数の名前は他のどの変数とも重複しないようにします。

本書のこれ以降の内容では、このような適切な項の選択を暗黙的に行うものとします。

\begin{exercise}

以下の $\lambda$ 項の変数名を適切に付け替えよ。

\[
  (\lambda f . \lambda x . f \, x) \, x
\]

\subparagraph{解答}

この項は、変数 $x$ が重複しています。束縛されている変数 $x$ を変数 $y$ に置き換えると、
\[
  (\lambda f . \lambda y . f \, y) \, x
\]
となります。

この項は、名前の重複している自由変数と束縛変数や、別々に束縛されている同じ名前の束縛変数を含みません。

この書き換えの $\alpha$ 同値関係としての正しさは、以下の証明木によって説明できます。

\[
  \infere{Eq-App}{
    (\lambda f . \lambda x . f \, x) \, x \alphaeq (\lambda f . \lambda y . f \, y) \, x
  }{
    \infere{Eq-Abs}{
      \lambda f . \lambda x . f \, x \alphaeq \lambda f . \lambda y . f \, y
    }{
      \infere{Eq-Alpha}{
        \lambda x . f \, x \alphaeq \lambda y . f \, y
      }{
        \text{項 $f \, x$ に変数 $y$ が自由出現しない}
      }
    } &
    \infere{Eq-Refl}{x \alphaeq x}{}
  }
\]

\end{exercise}

\section{$\beta$ 簡約}

$\lambda$ 計算の実行について考えてみましょう。

$\lambda$ 計算の実行の基本は、左辺が $\lambda$ 抽象になっている関数適用
($(\lambda x . e_1) \, e_2$ の形をしている $\lambda$ 項)を解く操作です。
この操作によって得られる項は、項 $e_1$ 中の自由変数 $x$ に項 $e_2$ を代入した項です。

項の中の関数適用を一箇所解く操作を、\textbf{$\beta$簡約 ($\beta$-reduction)} と呼びます。
$e_1$ を $\beta$ 簡約して $e_2$ にできるという二項関係を、
記号 $\betared$ を用いて $e_1 \betared e_2$ と表記します。
$\lambda$ 計算の $\beta$ 簡約を、図 \ref{fig:beta-reduction} のように定義します。

\begin{figure}[htbp]
  \[
    \infere{R-Beta}{
      (\lambda x . e_1) \, e_2 \betared e_1 [x := e_2]
    }{}
  \]
  \[
    \infere{R-App1}{
      e_1 \, e_2 \betared e_1' \, e_2
    }{
      e_1 \betared e_1'
    }
  \]
  \[
    \infere{R-App2}{
      e_1 \, e_2 \betared e_1 \, e_2'
    }{
      e_2 \betared e_2'
    }
  \]
  \[
    \infere{R-Abs}{
      \lambda x . e \betared \lambda x . e'
    }{
      e \betared e'
    }
  \]
  \caption{$\beta$ 簡約}
  \label{fig:beta-reduction}
\end{figure}

\textsc{R-Beta} 規則のように上が空になっている規則は、
無条件で下に書いてあることを認めて良いということを表しています。

$(\lambda x . e_1) e_2$ の形をしている項のみが、\textsc{R-Beta} 規則を用いて簡約できます。
このような形をしている項のことを、\textbf{$\beta$ 基 ($\beta$-redex)} と呼びます。

$\lambda$ 項の評価は $\beta$ 簡約の並びによって表現できます。

$\lambda$ 項の値としての同値関係は、
$\alpha$ 同値と $\beta$ 簡約の関係を含む最小の同値関係と定義できます。

\begin{exercise}

以下の $\beta$ 簡約が正しいことを示せ。

\[
  (\lambda x . x) \, f \, y \betared f \, y
\]

\subparagraph{解答}

これは、$(\lambda x . x) \, f$ の部分を簡約すれば良さそうです。$x [x := f] = f$ なので、
規則 \textsc{R-Beta} を使うと
\[
  \infere{R-Beta}{
    (\lambda x . x) \, f \betared f
  }{}
\]
と書けます。この部分式は全体から見ると関数適用の左辺なので、規則 \textsc{R-App1} を使うと
\[
  \infere{R-App1}{
    (\lambda x . x) \, f \, y \betared f \,
 y
  }{
    \infere{R-Beta}{
      (\lambda x . x) \, f \betared f
    }{}
  }
\]
と書けます。
このようにして、$(\lambda x . x) \, f \, y \betared f \, y$ が正しい $\beta$ 簡約であると示せます。

\end{exercise}

\begin{exercise}

以下の $\lambda$ 項を簡約せよ。

\[
  (\lambda x . x \, x) \, (\lambda x . x \, x)
\]

\subparagraph{解答}

この項の部分項は、外側の関数適用のみが $\beta$ 基となっています。
$x \, x [x := \lambda x . x \, x] = (\lambda x . x \, x) \, (\lambda x . x \, x)$ なので、
\[
  \infere{R-Beta}{
    (\lambda x . x \, x) \, (\lambda x . x \, x) \betared
    (\lambda x . x \, x) \, (\lambda x . x \, x)
  }{}
\]
となります。

しかし、簡約した結果として自分自身と同じ項が作れてしまいました。
このことから、この項は無限長の $\beta$ 簡約列を作ることができ、評価が停止しないと分かります。

この例以外にも、無限長の $\beta$ 簡約列を作れる $\lambda$ 項は無数に存在しています。

\end{exercise}

\section{型付き $\lambda$ 計算}

前の練習問題のように、無限長の $\beta$ 簡約列を作れるような $\lambda$ 項を作れます。

また、純粋な $\lambda$ 計算の範囲の話ではありませんが、定数値や定数値に対する演算を含む計算体系では、
特定の種類の定数値のみを対象とできる演算を考えたい場合があります。例えば、
\[
  \text{true} + 1
\]
は加算が数値に対する二項演算であって欲しいとすると、
真偽値 true と整数値 $1$ に対して適用されているため意図に反した計算であると言えます。

このような、停止しない計算や意図に反する計算を「正しくない計算」と見做し、正しくない計算を
表している項を排除する仕組みが\textbf{型付き $\lambda$ 計算 (typed lambda calculus)} です。
\footnote{
ただし、型付き $\lambda$ 計算から派生したプログラミング言語のほとんどは
プログラムが停止しないことや失敗することを許容しています。
しかし、その多くは型によってプログラムの間違いのほとんどを検出し、排除することに成功しています。}

型付き $\lambda$ 計算における\textbf{型 (type)} とは、とある項に対して適用可能な操作の表現です。

\subsection{単純型付き $\lambda$ 計算}

\textbf{単純型付き $\lambda$ 計算 (simply typed lambda calculus, $\lambda^\to$)} は、
型付きラムダ計算の体系の一つです。

単純型付き $\lambda$ 計算の型を、図\ref{fig:stlc-type}のように帰納的に定義します。

\begin{figure}[htbp]
  \begin{align*}
    t & \bnfcce  \alpha    && \text{(型変数)} \\
      & \bnfvert (t \to t) && \text{(関数の型)}
  \end{align*}
  \caption{単純型付き $\lambda$ 計算の型}
  \label{fig:stlc-type}
\end{figure}

このように、型変数と関数の型だけからなる表現を単純型付き $\lambda$ 計算の型とします。
型は同じ型変数を使うことで同じ型であることを表し、
$(t_1 \to t_2)$ は型 $t_1$ の値から型 $t_2$ の値への関数を表します。

例えば、$\lambda x. x$ (恒等関数)は $(\alpha \to \alpha)$ という型を持ちます。

$\lambda$ 項の表記と同様に、以下のような省略のための規則を導入します。

\begin{enumerate}
  \item 型 $t$ の部分 ($t_1 \to (t_2 \to t_3))$ は $t_1 \to t_2 \to t_3$ と省略できます。
  \item 型 $t$ が $(\dots)$ のように左右を対応する括弧で囲まれているとき、その括弧を省略できます。
\end{enumerate}

一番目の規則を繰り返し適用することで、部分 $((\dots (t_1 \to t_2) \to \dots) \to t_n)$ を
$(t_1 \to t_2 \to \dots \to t_n)$ と省略できます。
カリー化された多変数関数の型は、このような形を持ちます。

\begin{note}
ここまでの内容で、省略表記であることが明らかでない型の表記は省略されていてはならない。
\end{note}

\subsection{型判定}

単純型付き $\lambda$ 計算では、
項に正しく型付けできるということを\textbf{型判定 (type judgement)} で表します。

型環境 $\Gamma$ の下で項 $e$ の型が $t$ である、ということを
\[ \Gamma \vdash e : t \]
と表記します。

型環境 $\Gamma$ は、変数名と型のペアの集合($x_1 : t_1, x_2 : t_2, \dots, x_n : t_n$)です。
これは型を決定する上での文脈であり、変数の型はここから引いてくることになります。

型判定を型環境、項、型の間の関係とみなし、\textbf{型付け関係 (typing relation)} と呼ぶこともあります。

型環境 $\Gamma$ に「変数 $x$ の型が $t$ である」
という情報が含まれていることを $x : t \in \Gamma$ と表記します。
また、空の型環境を $\emptyset$ と表記し、型環境 $\Gamma$ に
「変数 $y$ の型が $t$ である」ということを追加した環境を $\Gamma , y : t$ と表記します。
型環境に関する規則を、図\ref{fig:stlc-type-environment}のように定義します。

\begin{figure}[htbp]
  \[
    \infere{T-Env1}{
      x : t \in \Gamma , x : t
    }{}
  \]
  \[
    \infere{T-Env2}{
      x : t_1 \in \Gamma , y : t_2
    }{
      x \neq y && x : t_1 \in \Gamma
    }
  \]
  \caption{型環境}
  \label{fig:stlc-type-environment}
\end{figure}

規則 \textsc{T-Env1} は、
型環境 $\Gamma, x : t$ に「変数 $x$ の型が $t$ である」が無条件に含まれる、と読みます。

規則 \textsc{T-Env2} は、
型環境 $\Gamma$ に「変数 $x$ の型が $t_1$ である」が含まれていてかつ $x \neq y$ であれば、
型環境 $\Gamma, y : t_2$ に「変数 $x$ の型が $t_1$ である」が含まれる、と読みます。

これらの規則は、単純に型環境に特定の変数と型のペアが含まれているということを
表す規則でないということに注意してください。
例えば、$x : t_1 \in \emptyset, x : t_1, y : t_2$ は成立しますが、
$x : t_2 \in \emptyset, x : t_1, x : t_2$ は成立しません。

なぜこうなっているのかというと、$\lambda x . \lambda x . x$ などの項における末尾の変数 $x$ は、
内側の $\lambda$ 抽象の $x$ であって外側の $\lambda$ 抽象の $x$ ではないからです。
型環境は後から追加された情報ほど深いスコープの変数になっているので、
最後に追加された同じ名前の変数のみを取れるような規則となっています。

単純型付き $\lambda$ 計算の型判定を、図\ref{fig:stlc-type-judgement}のように定義します。
このルールは、$\lambda$ 計算の項の種類一つに対してルールが一つであるということに注意してください。

\begin{figure}[htbp]
  \[
    \infere{T-Var}{
      \Gamma \vdash x : t
    }{
      x : t \in \Gamma
    }
  \]
  \[
    \infere{T-App}{
      \Gamma \vdash e_1 \, e_2 : t_2
    }{
      \Gamma \vdash e_1 : t_1 \to t_2 &
      \Gamma \vdash e_2 : t_1
    }
  \]
  \[
    \infere{T-Abs}{
      \Gamma \vdash \lambda x . e : t_1 \to t_2
    }
    {
      \Gamma, x : t_1 \vdash e : t_2
    }
  \]
  \caption{単純型付き $\lambda$ 計算の型判定}
  \label{fig:stlc-type-judgement}
\end{figure}

規則 \textsc{T-Var} は、型環境 $\Gamma$ に「変数 $x$ の型が $t$ である」が含まれていれば、
型環境 $\Gamma$ の下で式 $x$ の型は $t$ である、と読みます。

規則 \textsc{T-App} は、
型環境 $\Gamma$ の下で式 $e_1$ の型が $t_1 \to t_2$ かつ $e_2$ の型が $t_1$ であれば、
型環境 $\Gamma$ の下で式 $e_1 ~ e_2$ の型は $t_2$ である、と読みます。

規則 \textsc{T-Abs} は、
型環境 $\Gamma , x : t_1$ の下で式 $e$ の型が $t_2$ であれば、
型環境 $\Gamma$ の下で式 $\lambda x . e$ の型が $t_1 \to t_2$ である、と読みます。

また、定数などを含む $\lambda$ 項に型を付ける場合には、
それらの定数や関連する演算に関しての規則を追加する必要があります。
例えば、論理値(bool 型)と整数値(int 型)と整数値の加算を加えた定義は、
項に論理値と整数値を加え、型に bool 型と int 型を加えた上で、
図\ref{fig:stlc-type-judgement-constants}の型判定規則を追加します。

\begin{figure}[htbp]
  \[
    \infere{T-Bool}{
      \Gamma \vdash c : \text{bool}
    }{
      c \text{は論理値}
    }
  \]
  \[
    \infere{T-Int}{
      \Gamma \vdash c : \text{int}
    }{
      c \text{は整数値}
    }
  \]
  \[
    \infere{T-Plus}{
      \Gamma \vdash e_1 + e_2 : \text{int}
    }{
      \Gamma \vdash e_1 : \text{int} &
      \Gamma \vdash e_2 : \text{int}
    }
  \]
  \caption{定数に関する追加の型判定規則}
  \label{fig:stlc-type-judgement-constants}
\end{figure}

これらの規則は簡単なので、練習も兼ねて自分で読んでみてください。

\subsection{型を付ける}

ここまでに出てきた規則を利用して、実際に $\lambda$ 項に型を付けてみましょう。

\subsubsection{例1 : $\lambda x . x + 1$}

例として、項 $\lambda x . x + 1$ に型を付けてみます。
項の外側から、即ち型判定木の下の方から構築していきましょう。一番外側は $\lambda$ 抽象なので、
\[
  \infere{T-Abs}{
    \emptyset \vdash \lambda x . x + 1 : ?_0 \to ?_1
  }{
    \emptyset , x : ?_0 \vdash x + 1 : ?_1
  }
\]
と書けます。

ただし、ここでの記号 $?$ はまだ確定していない型を表現しています。
同じ番号を振ることで同じ型であることを表現しています。

次に出てきたのは加算なので、
\[
  \infere{T-Abs}{
    \emptyset \vdash \lambda x . x + 1 : ?_0 \to \text{int}
  }{
    \infere{T-Plus}{
      \emptyset , x : ?_0 \vdash x + 1 : \text{int}
    }{
      \emptyset , x : ?_0 \vdash x : \text{int} &
      \emptyset , x : ?_0 \vdash 1 : \text{int}
    }
  }
\]
と書けます。規則 \textsc{T-Plus} によって $?_1$ が int 型に置き換えられたことに注意してください。

次に、左側の項 $x$ の型判定を埋めましょう。規則 \textsc{T-Var} と規則 \textsc{T-Env1} を使うと、
\[
  \infere{T-Abs}{
    \emptyset \vdash \lambda x . x + 1 : \text{int} \to \text{int}
  }{
    \infere{T-Plus}{
      \emptyset , x : \text{int} \vdash x + 1 : \text{int}
    }{
      \infere{T-Var}{
        \emptyset , x : \text{int} \vdash x : \text{int}
      }{
        \infere{T-Env1}{
          x : \text{int} \in \emptyset , x : \text{int}
           }{}
      } &
      \emptyset , x : \text{int} \vdash 1 : \text{int}
    }
  }
\]
と書けます。ここでも、規則 \textsc{T-Env1} の制約により $?_0$ が int 型に置き換わりました。

最後に、定数1の型判定を埋めます。規則 \textsc{T-Int} を使って、
\[
  \infere{T-Abs}{
    \emptyset \vdash \lambda x . x + 1 : \text{int} \to \text{int}
  }{
    \infere{T-Plus}{
      \emptyset , x : \text{int} \vdash x + 1 : \text{int}
    }{
      \infere{T-Var}{
        \emptyset , x : \text{int} \vdash x : \text{int}
      }{
        \infere{T-Env1}{
          x : \text{int} \in \emptyset , x : \text{int}
           }{}
      } &
      \infere{T-Int}{
        \emptyset , x : \text{int} \vdash 1 : \text{int}
      }{
        1 \text{は整数値}
      }
    }
  }
\]
と書けます。これで型判定全体が作れました。

このようにして、
$\lambda x . x + 1$ の型が $\text{int} \to \text{int}$ であることの理由付けができます。

\subsubsection{例2 : $\lambda x . \lambda y . x$}

もう一つの例として、項 $\lambda x . \lambda y . x$ に型を付けてみます。
まず、項全体は $\lambda$ 抽象なので、
\[
  \infere{T-Abs}{
    \emptyset \vdash \lambda x . \lambda y . x : ?_0 \to ?_1
  }{
    \emptyset , x : ?_0 \vdash \lambda y . x : ?_1
  }
\]
と書けます。

次に現れているのも $\lambda$ 抽象なので、
\[
  \infere{T-Abs}{
    \emptyset \vdash \lambda x . \lambda y . x : ?_0 \to ?_2 \to ?_3
  }{
    \infere{T-Abs}{
      \emptyset , x : ?_0 \vdash \lambda y . x : ?_2 \to ?_3
    }{
      \emptyset , x : ?_0 , y : ?_2 \vdash x : ?_3
    }
  }
\]
と書けます。規則 \textsc{T-Abs} の制約により、$?_1$ が $?_2 \to ?_3$ に置き換わりました。

次に現れているのは変数なので、
\[
  \infere{T-Abs}{
    \emptyset \vdash \lambda x . \lambda y . x : ?_0 \to ?_2 \to ?_0
  }{
    \infere{T-Abs}{
      \emptyset , x : ?_0 \vdash \lambda y . x : ?_2 \to ?_0
    }{
      \infere{T-Var}{
          \emptyset , x : ?_0 , y : ?_2 \vdash x : ?_0
      }{
        \infere{T-Env2}{
          x : ?_0 \in \emptyset , x : ?_0 , y : ?_2
         }{
          x \neq y &
          \infere{T-Env1}{
            x : ?_0 \in \emptyset , x : ?_0
          }{}
        }
      }
    }
  }
\]
と書けます。規則 \textsc{T-Env} の制約により $?_0$ と $?_3$ が同一であると分かったため、
$?_3$ を $?_0$ で置き換えました。ただし、これは $?_0$ を $?_3$ で置き換えても構いません。

これで一応型は付けられたように見えますが、記号 $?$ がそのまま残ってしまいました。
これらは、とりあえず型変数で表現することにします。
今の例であれば、$\alpha \to \beta \to \alpha$ のように書きます。

この推論後に残ってしまう $?$ については、また3章で別の形で出てきます。

\begin{exercise}

以下の項が単純型付き $\lambda$ 計算において型付け不可能であることを示せ。

\[
  \text{true} + 1
\]

\subparagraph{解答}

これは節の最初に出てきた意図に反する項の例です。このような項は型システムによって排除されるはずです。
本当に型付け不可能なのか確かめてみましょう。

まず、項全体は加算になっているので、
\[
  \infere{T-Plus}{
    \emptyset \vdash \text{true} + 1 : \text{int}
  }{
    \emptyset \vdash \text{true} : \text{int} &
    \emptyset \vdash 1 : \text{int}
  }
\]
と書けます。

しかし、ここで $\emptyset \vdash \text{true} : \text{int}$ の部分に注目すると、
このような型判定を作れる規則は存在しません。

\end{exercise}

\begin{exercise}

以下の項が単純型付き $\lambda$ 計算において型付け不可能であることを示せ。

\[
  (\lambda x . x \, x) \, (\lambda x . x \, x)
\]

\subparagraph{解答}

これは無限長の $\beta$ 簡約列を作れる $\lambda$ 項の例です。
本当に型付け不可能なのか確かめてみましょう。

まず、項全体が関数適用になっているので、
\[
  \infere{T-App}{
    \emptyset \vdash (\lambda x . x \, x) \, (\lambda x . x \, x) : ?_1
  }{
    \emptyset \vdash \lambda x . x \, x : ?_0 \to ?_1 &
    \emptyset \vdash \lambda x . x \, x : ?_0
  }
\]
と書けます。

次に、左側の $\lambda$ 抽象に注目すると、
\[
  \infere{T-App}{
    \emptyset \vdash (\lambda x . x \, x) \, (\lambda x . x \, x) : ?_1
  }{
    \infere{T-Abs}{
      \emptyset \vdash \lambda x . x \, x : ?_0 \to ?_1
    }{
      \{x : ?_0\} \vdash x \, x : ?_1
    } &
    \emptyset \vdash \lambda x . x \, x : ?_0
  }
\]
と書けます。ここで、$\{x : ?_0\}$ というのは $\emptyset, x : ?_0$ の簡略表記です。
これ以降も、$\emptyset, x_1 : t_1, \dots, x_n : t_n$ を $\{x_1 : t_1, \dots, x_n : t_n\}$
と表記することがあります。

次に、$x \, x$ の関数適用に注目すると、
\[
  \infere{T-App}{
    \emptyset \vdash (\lambda x . x \, x) \, (\lambda x . x \, x) : ?_1
  }{
    \infere{T-Abs}{
      \emptyset \vdash \lambda x . x \, x : ?_0 \to ?_1
    }{
      \infere{T-App}{
        \{x : ?_0\} \vdash x \, x : ?_1
      }{
        \{x : ?_0\} \vdash x : ?_2 \to ?_1 &
        \{x : ?_0\} \vdash x : ?_2
      }
    } &
    \emptyset \vdash \lambda x . x \, x : ?_0
  }
\]
と書けます。

次に、二つの変数 $x$ に注目すると、
\[
  \infere{T-App}{
    \emptyset \vdash (\lambda x . x \, x) \, (\lambda x . x \, x) : ?_1
  }{
    \infere{T-Abs}{
      \emptyset \vdash \lambda x . x \, x : ?_0 \to ?_1
    }{
      \infere{T-App}{
        \{x : ?_0\} \vdash x \, x : ?_1
      }{
          \infere{T-Var}{
          \{x : ?_0\} \vdash x : ?_0 \to ?_1
         }{
          x : ?_0 \to ?_1 \in \{x : ?_0\}
        } &
         \infere{T-Var}{
          \{x : ?_0\} \vdash x : ?_0
        }{
          \infere{T-Env1}{
            x : ?_0 \in \{x : ?_0\}
          }{}
        }
      }
    } &
    \emptyset \vdash \lambda x . x \, x : ?_0
  }
\]
と書けます。規則 \textsc{T-Var} 内の制約により $?_2$ が $?_0$ に置き換わりました。

しかし、ここで $x : ?_0 \to ?_1 \in \{x : ?_0\}$ の部分がまだ書かれていません。
この形を見る限りでは規則 \textsc{T-Env1} が適用されるように見えますが、
もしそうだとすると $?_0 = ?_0 \to ?_1$ であるような型 $?_0$ がなくてはいけません。
しかしそれは $((\dots \to ?_1) \to ?_1) \to ?_1$ のような無限長の型になってしまいます。
\footnote{型推論において、このような無限長の型が作られないことを保証し、
型推論が停止しないことを防ぐための仕組みを出現検査(occurs check)と言います。
これは2章で説明します。}

型の定義は帰納的であるため、このような無限長の型を作ることはできません。
このことから、項 $(\lambda x . x \, x) \, (\lambda x . x \, x)$ は型付けできないと分かります。

\subparagraph{別解}

帰納法により、「項が型付けできればその項の任意の部分は型付け可能」、
また、その対偶「とある項の部分が型付け不可能であれば全体も型付け不可能」が証明できます。

$\lambda x . x \, x$ 自体が型付け不可能なので、この二つの事実から
$(\lambda x . x \, x) \, (\lambda x . x \, x)$ が型付け不可能であると言えます。

\end{exercise}

%\section{型検査}



\section{まとめ}

1章では、型推論を始める前に、その前提となる知識を一通り解説しています。主に三つの物事を説明しました。

一つ目は、型無し $\lambda$ 計算の基礎です。
ここでは、$\lambda$ 項の定義、$\alpha$ 変換とそれによる変数名の適切な選択、
$\beta$ 簡約について説明しました。
これによって、型無し $\lambda$ 計算の定義と計算体系としての意味を与えています。
また、項としての同値関係と値としての同値関係を規定しています。

二つ目は、型付き $\lambda$ 計算の概要と、
$\lambda$ 計算の型付けの体系の一種である単純型付き $\lambda$ 計算の定義です。
これによって、型無し $\lambda$ 計算で発生し得る計算として正しくない項を排除し、
型によって項に意味を与えられました。

三つ目は、型の付けられていない単純型付き $\lambda$ 計算の項に型を付ける方法です。
これによって、次の章から始まる型推論に相当する計算を、人の手によってできるようになりました。



\chapter{単純型付き $\lambda$ 計算の型推論}


この章から、本題である所の型推論について説明していきます。

\section{型推論とは}

型推論とは、項からその項の型を計算することです。

最近の多くの言語処理系はプログラムにおける型を計算するための型推論の機能を持っており、
一部のプログラマにとっても身近かつ必要不可欠なものとなっているでしょう。

単純型付き $\lambda$ 計算の型推論が何をするのか、第1章で説明した型判定を使って説明してみましょう。

仮に、型推論を任意の型環境 $\Gamma$ と任意の $\lambda$ 項 $e$ に対し、$\Gamma \vdash e : t$
であるような型 $t$ を計算する操作としてみましょう。

しかし、これには以下に示した二つの問題点があります。

\begin{enumerate}
  \item 型が付けられない $\lambda$ 項に関しては、解が存在しません。
  \item 無数の解が存在するケースがあります。例えば、型環境 $\emptyset$ の下で項 $\lambda x . x$ は
        $\alpha \to \alpha$ や $(\alpha \to \beta) \to \alpha \to \beta$ など、
        $?_0 \to ?_0$ の形を持つ型全てで型付けできます。
\end{enumerate}

まず、一番目は解があるかないかを判定し、解がある場合のみ型を計算するとしましょう。

二番目に関しては、最も広い範囲を指す型の表現、つまり $\lambda x . x$ に対する $?_0 \to ?_0$
のような型を計算するとしましょう。

以上をまとめて、単純型付き $\lambda$ 計算の型推論を以下のように定義します。

\begin{definition}[単純型付き $\lambda$ 計算の型推論]
単純型付き $\lambda$ 計算の型推論は、任意の型環境 $\Gamma$ と任意の $\lambda$ 項 $e$ に対し、
$\Gamma \vdash e : t$ であるような型 $t$ が存在するかを計算する。
また、もしそのような型が存在するのであればその $t$ のうち最も広い範囲の型を計算する。
\end{definition}

この定義のような型推論は、決定可能であることが知られています。
型推論のアルゴリズムについて考えていきましょう。

\section{アルゴリズム}

型推論のアルゴリズムは、大きく分けて型に関する方程式の生成と、
その方程式を解く二つの部分によって構成されています。

ここでいう方程式とは、型の等式集合のことです。第1章で出てきた実際に型を付けてみる例では、
とある型ととある型が同一であるという等式はその場で解いてしまっていましたが、
ここではまずこの等式集合を生成し、それを後から解くという方針を取ります。

\subsection{例}

型推論のアルゴリズムの説明の前に、どうすれば型推論が実現できるのか、
特定の例を持ち出して考えてみましょう。

型環境 $\emptyset$ の下で $\lambda f . f \, (f \, (\lambda x . x))$
という $\lambda$ 項がどのような型を持つか、という例を用います。まずは型の等式集合を作ってみましょう。

また、今までは証明木を構築する時に木の全体を書いていましたが、
今回は後から $?$ が書き変わることがないので、その時点で見ている部分のみを書くことにしましょう。

まず、外側は $\lambda$ 抽象なので、
\[
  \infere{T-Abs}{
    \emptyset \vdash \lambda f . f \, (f \, (\lambda x . x)) : ?_0
  }{
    \{f : ?_1\} \vdash f \, (f \, (\lambda x . x)) : ?_2
  }
\]
となります。規則 \textsc{T-Abs} により、$?_0 = ?_1 \to ?_2$ という制約が追加されます。

次は関数適用なので、
\[
  \infere{T-App}{
    \{f : ?_1\} \vdash f \, (f \, (\lambda x . x)) : ?_2
  }{
    \{f : ?_1\} \vdash f : ?_3 \to ?_2 &
    \{f : ?_1\} \vdash f \, (\lambda x . x) : ?_3
  }
\]
となります。

次は右側の関数適用に注目すると、
\[
  \infere{T-App}{
    \{f : ?_1\} \vdash f \, (\lambda x . x) : ?_3
  }{
    \{f : ?_1\} \vdash f : ?_4 \to ?_3 &
    \{f : ?_1\} \vdash \lambda x . x : ?_4
  }
\]
となります。

次は右側の $\lambda$ 抽象に注目すると、
\[
  \infere{T-Abs}{
    \{f : ?_1\} \vdash \lambda x . x : ?_4
  }{
    \{f : ?_1, x : ?_5\} \vdash x : ?_6
  }
\]
となります。規則 \textsc{T-Abs} により、$?_4 = ?_5 \to ?_6$ という制約が追加されます。

残りは変数です。
まずは $\{f : ?_1\} \vdash f : ?_3 \to ?_2$ を解くと、
\[
  \infere{T-Var}{
    \{f : ?_1\} \vdash f : ?_3 \to ?_2
  }{
    \infere{T-Env1}{
      f : ?_3 \to ?_2 \in \{f : ?_1\}
    }{}
  }
\]
となります。規則 \textsc{T-Env1} により、$?_3 \to ?_2 = ?_1$ という制約が追加されます。

$\{f : ?_1\} \vdash f : ?_4 \to ?_3$ を解くと、
\[
  \infere{T-Var}{
    \{f : ?_1\} \vdash f : ?_4 \to ?_3
  }{
    \infere{T-Env1}{
      f : ?_4 \to ?_3 \in \{f : ?_1\}
    }{}
  }
\]
となります。規則 \textsc{T-Env1} により、$?_4 \to ?_3 = ?_1$ という制約が追加されます。

$\{f : ?_1, x : ?_5\} \vdash x : ?_6$ を解くと、
\[
  \infere{T-Var}{
    \{f : ?_1, x : ?_5\} \vdash x : ?_6
  }{
    \infere{T-Env1}{
      x : ?_6 \in \{f : ?_1, x : ?_5\}
    }{}
  }
\]
となります。規則 \textsc{T-Env1} により、$?_6 = ?_5$ という制約が追加されます。

これで、方程式の生成ができました。等式を列挙すると
\begin{itemize}
  \item $?_0 = ?_1 \to ?_2$
  \item $?_4 = ?_5 \to ?_6$
  \item $?_3 \to ?_2 = ?_1$
  \item $?_4 \to ?_3 = ?_1$
  \item $?_6 = ?_5$
\end{itemize}
となっています。

方程式を解いた結果は $?_n$ とその実体の対応という形になるとしましょう。
これを\textbf{代入 (substitution)} といいます。
これを $[?_a := t_a, ?_b := t_b , \dots]$ のように表記します。
項に出現する特定の自由変数を全て特定の項で置き換える操作も代入と呼ばれますが、
別のものであるということに注意してください。

方程式を解く操作は、等式集合と代入のペアを適切に変換することによって実現します。
最初の状態は、与えられた等式集合と空の代入のペアです。

では実際に書き換えを行ってみましょう。以下に書き換えの手順を示します。

\begin{enumerate}
  \item
    \begin{description}
      \item[等式集合]
        $\{?_0 = ?_1 \to ?_2, ?_4 = ?_5 \to ?_6, ?_3 \to ?_2 = ?_1, ?_4 \to ?_3 = ?_1, ?_6 = ?_5\}$
      \item[代入]
        $[]$
    \end{description}
    等式 $?_0 = ?_1 \to ?_2$ を解きます。
    一方が変数なので、等式集合と代入の $?_0$ を全て $?_1 \to ?_2$ に置き換え、
    代入に $?_0 := ?_1 \to ?_2$ を追加します。
  \item
    \begin{description}
      \item[等式集合]
        $\{?_4 = ?_5 \to ?_6, ?_3 \to ?_2 = ?_1, ?_4 \to ?_3 = ?_1, ?_6 = ?_5\}$
      \item[代入]
        $[?_0 := ?_1 \to ?_2]$
    \end{description}
    等式 $?_4 = ?_5 \to ?_6$ を解きます。
    一方が変数なので、等式集合と代入の $?_4$ を全て $?_5 \to ?_6$ に置き換え、
    代入に $?_4 := ?_5 \to ?_6$ を追加します。
  \item
    \begin{description}
      \item[等式集合]
        $\{?_3 \to ?_2 = ?_1, (?_5 \to ?_6) \to ?_3 = ?_1, ?_6 = ?_5\}$
      \item[代入]
        $[?_0 := ?_1 \to ?_2, ?_4 := ?_5 \to ?_6]$
    \end{description}
    等式 $?_3 \to ?_2 = ?_1$ を解きます。
    一方が変数なので、等式集合と代入の $?_1$ を全て $?_3 \to ?_2$ に置き換え、
    代入に $?_1 := ?_3 \to ?_2$ を追加します。
  \item
    \begin{description}
      \item[等式集合]
        $\{(?_5 \to ?_6) \to ?_3 = ?_3 \to ?_2, ?_6 = ?_5\}$
      \item[代入]
        $[?_0 := (?_3 \to ?_2) \to ?_2, ?_1 := ?_3 \to ?_2, ?_4 := ?_5 \to ?_6]$
    \end{description}
    等式 $(?_5 \to ?_6) \to ?_3 = ?_3 \to ?_2$ を解きます。
    どちらも関数の型なので、$?_5 \to ?_6 = ?_3$ と $?_3 = ?_2$ の二つの等式に分解します。
  \item
    \begin{description}
      \item[等式集合]
        $\{?_5 \to ?_6 = ?_3, ?_3 = ?_2, ?_6 = ?_5\}$
      \item[代入]
        $[?_0 := (?_3 \to ?_2) \to ?_2, ?_1 := ?_3 \to ?_2, ?_4 := ?_5 \to ?_6]$
    \end{description}
    等式 $?_5 \to ?_6 = ?_3$ を解きます。
    一方が変数なので、等式集合と代入の $?_3$ を全て $?_5 \to ?_6$ に置き換え、
    代入に $?_3 := ?_5 \to ?_6$ を追加します。
  \item
    \begin{description}
      \item[等式集合]
        $\{?_5 \to ?_6 = ?_2, ?_6 = ?_5\}$
      \item[代入]
        $[?_0 := ((?_5 \to ?_6) \to ?_2) \to ?_2, ?_1 := (?_5 \to ?_6) \to ?_2,
          ?_3 := ?_5 \to ?_6, ?_4 := ?_5 \to ?_6]$
    \end{description}
    等式 $?_5 \to ?_6 = ?_2$ を解きます。
    一方が変数なので、等式集合と代入の $?_2$ を全て $?_5 \to ?_6$ に置き換え、
    代入に $?_2 := ?_5 \to ?_6$ を追加します。
  \item
    \begin{description}
      \item[等式集合]
        $\{?_6 = ?_5\}$
      \item[代入]
        $[?_0 := ((?_5 \to ?_6) \to ?_5 \to ?_6) \to ?_5 \to ?_6,
          ?_1 := (?_5 \to ?_6) \to ?_5 \to ?_6, ?_2 := ?_5 \to ?_6, ?_3 := ?_5 \to ?_6,
          ?_4 := ?_5 \to ?_6]$
    \end{description}
    等式 $?_6 = ?_5$ を解きます。
    一方が変数なので、等式集合と代入の $?_6$ を全て $?_5$ に置き換え、
    代入に $?_6 := ?_5$ を追加します。
  \item
    \begin{description}
      \item[等式集合]
        $\{\}$
      \item[代入]
        $[?_0 := ((?_5 \to ?_5) \to ?_5 \to ?_5) \to ?_5 \to ?_5,
          ?_1 := (?_5 \to ?_5) \to ?_5 \to ?_5, ?_2 := ?_5 \to ?_5, ?_3 := ?_5 \to ?_5,
          ?_4 := ?_5 \to ?_5, ?_6 := ?_5]$
    \end{description}
    等式集合が空になったので、これで方程式が解けました。
\end{enumerate}

目的の型は $\emptyset \vdash \lambda f . f \, (f \, (\lambda x . x)) : ?_0$ とあることから分かる通り
$?_0$ なので、代入から $?_0$ に対応する型を取れば良いはずです。
よって、型環境 $\emptyset$ の下で項 $\lambda f . f \, (f \, (\lambda x . x))$ は
$((\alpha \to \alpha) \to \alpha \to \alpha) \to \alpha \to \alpha$ で型付けできると分かります。
また、他の型で型付けできるとしても、それはこの型の $\alpha$ に別の型を代入した形を持ちます。

単純型付き $\lambda$ 計算の型推論は、このような方法で実現できます。
では、方程式の生成と制約解消系について、その詳細を順番に見ていきましょう。

\subsection{方程式の生成}

方程式の生成アルゴリズムを考えていきます。

まず、前提として今まで使ってきた $?_n$ のような
ユニークな型上の変数を無制限に作り出す仕組みがあると仮定しましょう。
また、この仕組みによって作られた型変数のことを\textbf{フレッシュな型変数}と呼びます。

方程式を生成するアルゴリズムの名前を $\mathcal C$ としましょう
\footnote{constraints の c から取りました。}。
このアルゴリズムは、型環境 $\Gamma$ と項 $e$ を取り、方程式と項 $e$ の型を返すものとします。
また、項 $e$ の自由変数とその型は型環境 $\Gamma$ に含まれていなければならないものとします。

まず、$\mathcal C$ が型環境 $\Gamma$ と変数の項 $x$ を取る場合を考えます。
この場合は方程式は空で、型環境 $\Gamma$ から $x$ に対応する型を探して返します。

次に、$\mathcal C$ が型環境 $\Gamma$ と関数適用の項 $e_1 \, e_2$ を取る場合を考えます。
この場合はまず、$e_1$ と $e_2$ それぞれに関して $\mathcal C$ を適用します。
適用した結果が $(c_1, t_1) = \mathcal{C}(\Gamma, e_1)$ と $(c_2, t_2) = \mathcal{C}(\Gamma, e_2)$
のようになるとします。

さて、$e_1 \, e_2$ の型は何になるでしょうか。
$t_1$ でも $t_2$ でもないことは明らかなので、フレッシュな型変数 $t_3$ を作り、これを結果としましょう。
$e_1 : t_1$ は $e_2 : t_2$ を引数に取り $e_1 \, e_2 : t_3$ を返すので、
この三つの型の間に $t_1 = t_2 \to t_3$ という等式が成り立つはずです。これが新しい制約です。
方程式全体は単純に足せば良いので、$c_1 \cup c_2 \cup \{t_1 = t_2 \to t_3\}$ と表せます。
\footnote{$A \cup B$ は集合 $A$ と集合 $B$ の和集合のことです。}

次に、$\mathcal C$ が型環境 $\Gamma$ と $\lambda$ 抽象の項 $\lambda x . e$ を取る場合を考えます。
この場合はまず、フレッシュな型変数 $t_1$ を作り、
型環境 $\Gamma, x : t_1$ と項 $e$ に計算 $\mathcal C$ を適用します。
適用した結果が $(c, t_2) = \mathcal{C}(\Gamma, x : t_1, e)$ のようになるとします。
とくに追加する制約はないので、結果の方程式は $c$、$\lambda x . e$ の型は $t_1 \to t_2$ となります。

以上をまとめると、方程式の生成アルゴリズム $\mathcal C$ は図\ref{fig:algorithm-c}のようになります。

\begin{figure}[htbp]
  \begin{align*}
    \mathcal{C}(\Gamma, x) &=
      \mathrm{if} ~ x \notin \mathit{dom}(\Gamma) ~
      \mathrm{then} ~ \mathit{failure} ~
      \mathrm{else} ~ (\emptyset, \Gamma(x)) \\
    \mathcal{C}(\Gamma, e_1 \, e_2) &=
    \begin{array}[t]{l}
      \mathrm{let}
        \begin{array}[t]{l}
          (c_1, t_1) = \mathcal{C}(\Gamma, e_1) \\
          (t_2, t_2) = \mathcal{C}(\Gamma, e_2) \\
          t_3 = \mathrm{fresh}
        \end{array} \\
      \mathrm{in} (c_1 \cup c_2 \cup \{t_1 = t_2 \to t_3\}, t_3)
    \end{array} \\
    \mathcal{C}(\Gamma, \lambda x . e) &=
    \begin{array}[t]{l}
      \mathrm{let}
        \begin{array}[t]{l}
          t_1 = \mathrm{fresh} \\
          (c, t_2) = \mathcal{C}((\Gamma, x : t_1), e)
        \end{array} \\
      \mathrm{in} (c, t_1 \to t_2)
    \end{array}
  \end{align*}
  \caption{方程式の生成アルゴリズム $\mathcal C$}
  \label{fig:algorithm-c}
\end{figure}

この再帰的なアルゴリズム $\mathcal C$ は、項に関する構造的帰納法を使って記述できます。

また、
自由変数とその型は全て型環境に含まれているということの検査を含んでいるということに注意してください。
もしこの点が守られていなければ、$\mathit{failure}$ が返ります。
また、$\mathit{dom}(\Gamma)$ は型環境 $\Gamma$ の中の変数全ての集合を表します。
$\Gamma(x)$ というのは型環境 $\Gamma$ から $x$ の型を取ってくることを表します。
$\mathrm{fresh}$ はフレッシュな型変数を表します。

定数などを含む単純型付き $\lambda$ 計算に対しても、
同様の方法で対応するルールに沿うように制約の計算方法を追加することで正しい方程式の生成ができます。

このアルゴリズムによって生成される方程式を解くことで、単純型付き $\lambda$ 計算の型推論ができます。

\subsection{制約解消系}

次に、アルゴリズム $\mathcal C$ で生成した方程式を解くアルゴリズムについて考えます。

方程式を解くアルゴリズムの名前を $\mathcal U$ としましょう
\footnote{unification の u から取りました。}。
この計算は、等式集合を取り、それを解いた結果として得られる型変数への代入を返すとしましょう。
このアルゴリズムにおいて、等式 $t_1 = t_2$ と $t_2 = t_1$ は同一のものであるとします。

また、$S = \mathcal{U}(E)$ であるなら、
$E$ 中の任意の等式 $t_1 = t_2$ について $t_1 S = t_2 S$ が成り立つものとします。
このようなアルゴリズムを、\textbf{単一化 (unification)}のアルゴリズムといいます。

この節の頭で示した例での方針に従い、等式集合と代入の組の適切な変形規則を作り、
その変形の繰り返しによって得られる代入を結果とします。

まず、等式集合が $t = t$ と残りの等式集合に分割できる場合を考えます。
任意の代入は $t = t$ を満たすと考えられるので、これは単に $t = t$ を取り除けば良いはずです。よって、
\[
  (E \cup \{t = t\}, S) \Longrightarrow (E, S)
\]
という変形ができます。

次に、等式集合が $\alpha = t$ と残りの等式集合に分割できる場合を考えます。
この場合は、残りの等式と代入に出現する $\alpha$ を全て $t$ に置き換え、
代入に $\alpha := t$ を追加すれば良いはずです。よって、
\[
  (E \cup \{\alpha = t\}, S) \Longrightarrow (E[\alpha := t], S[\alpha := t] \cup [\alpha := t])
\]
という変形ができます。

ただし、$t$ の中に $\alpha$ が出現している場合、
置き換えの結果としてまた $\alpha$ が出てきてしまうので、$\alpha \notin \mathit{FTV}(t)$
でなければなりません。$\mathit{FTV}(t)$ は型 $t$ 中の型変数の集合を意味します。
これは free type variables の頭文字を取っています。

もし仮に $\alpha \in \mathit{FTV}(t)$ かつ $t \neq \alpha$ であったとすると、
$t$ は無限長の型になってしまうはずです。
しかし、型は帰納的に定義されているので、このことからも解が存在しないと言えます。

次に、等式集合が $t_1 \to t_2 = t_1' \to t_2'$ と残りの等式集合に分割できる場合を考えます。
この等式は、$t_1 = t_1'$ と $t_2 = t_2'$ という二つの等式に分割できると考えられます。
よって、
\[
  (E \cup \{t_1 \to t_2 = t_1' \to t_2'\}, S) \Longrightarrow (E \cup \{t_1 = t_1', t_2 = t_2'\}, S)
\]
という変形ができます。

以上で、変形規則は全てです。
単一化アルゴリズムの変形規則 $\Longrightarrow$ を、図\ref{fig:unification-trans}のように定義します。

\begin{figure}[htbp]
  \begin{align*}
    (E \cup \{t = t\}, S)
      &\Longrightarrow (E, S) \\
    (E \cup \{\alpha = t\}, S)
      &\Longrightarrow (E[\alpha := t], S[\alpha := t] \cup [\alpha := t])
      &\text{($\alpha \notin \mathit{FTV(t)}$)}\\
    (E \cup \{t_1 \to t_2 = t_1' \to t_2'\}, S)
      &\Longrightarrow (E \cup \{t_1 = t_1', t_2 = t_2'\}, S)
  \end{align*}
  \caption{単一化アルゴリズムの変形規則}
  \label{fig:unification-trans}
\end{figure}

また、この変形を0回以上の任意の回数繰り返す変形を、$\stackrel{*}{\Longrightarrow}$ で表現します。
この変形を用いて、アルゴリズム $\mathcal U$ を図\ref{fig:algorithm-u} のように定義します。

\begin{figure}[htbp]
  \begin{align*}
    U(E) &=
    \begin{cases}
      S & \text{($(E, \{\}) \stackrel{*}{\Longrightarrow} (\{\}, S)$)} \\
      \mathit{failure} & \text{(上記以外)}
    \end{cases}
  \end{align*}
  \caption{単一化アルゴリズム $\mathcal U$}
  \label{fig:algorithm-u}
\end{figure}

このアルゴリズムの停止性は、容易には説明ができません。
念のため、どのような理由でこのアルゴリズムが停止するのか考えてみましょう。

解く問題そのものとも言える等式集合の大きさを論じるために、
等式集合に自然数を対応させる関数を二つ考えます。
一方は、等式集合 $E$ 中の型変数の数 $\mathit{var}(E)$ です
\footnote{「型変数の数」の意味は「型変数の出現の数」とは異なります。
複数の同じ型変数の出現を一つの型変数と数えます。}。
もう一方は、等式集合 $E$ 中の全てのシンボル($\to$ と $=$)の数 $\mathit{sym}(E)$ です。

一つ目の変換規則に注目すると、$\mathit{var}(E \cup \{t = t\}) \ge \mathit{var}(E)$ と
$\mathit{sym}(E \cup \{t = t\}) > \mathit{sym}(E)$ が成立します。
よって、この変換は型変数の数は変わらないか減少するかのどちらかであり、
シンボルの数は確実に減少すると言えます。

二つ目の変換規則に注目すると、
$\mathit{var}(E \cup \{\alpha = t\}) = \mathit{var}(E[\alpha := t])+1$ が成立します。
よって、この変換は型変数の数が確実に減少すると言えます。

三つ目の変換規則に注目すると、
$\mathit{var}(E \cup \{t_1 \to t_2 = t_1' \to t_2'\}) =
 \mathit{var}(E \cup \{t_1 = t_1', t_2 = t_2'\})$
と
$\mathit{sym}(E \cup \{t_1 \to t_2 = t_1' \to t_2'\}) =
 \mathit{sym}(E \cup \{t_1 = t_1', t_2 = t_2'\})+1$
が成立します。
よって、この変換は型変数の数は変わらず、シンボルの数が確実に減少すると言えます。

さて、これらの三つの規則に関する性質を見ると、
それぞれ型変数の数とシンボルの数の辞書式順序に関して等式集合が減少しています。
このことから、単一化のアルゴリズムは停止すると言えます。

また、このような単一化のアルゴリズムはどのような順序で制約を解いたとしても、
\begin{itemize}
  \item 解が存在しない場合かつその時に限りエラーを報告する
  \item 解が存在する場合は正しい解を返す
  \item 解が存在する場合の結果は常に最も一般的な解である
\end{itemize}
の三点を守ることが保証されています。

単一化アルゴリズムが最も一般的な解を返すという性質から、
型推論が最も一般的な型を推論することが説明できます。

\section{実装}

ここまでで説明したアルゴリズムを OCaml で実装して、動かしてみましょう。

まず、core ライブラリを使うために、\texttt{Core.Std} モジュールを開きます。

\begin{lstlisting}
open Core.Std
\end{lstlisting}

このようにすることで、
core ライブラリのモジュール名が標準ライブラリの拡張のように見えるようになります。

例えば、core ライブラリに含まれる \texttt{Core.Std.List} モジュールが、
これによって \texttt{List} という名前で参照できるようになります。

\subsection{データ型}

型推論を書く前に、その対象となる項と型を表現するデータ型を定義しましょう。

まずは、項のデータ型を定義します。項の定義は図\ref{fig:lambda-term}の通りです。
これを定義するとリスト\ref{list:ocaml-def-term}のようになります。

\begin{lstlisting}[caption=項の定義, label=list:ocaml-def-term]
type term
  = EVar of string
  | EApp of (term * term)
  | EAbs of (string * term)
\end{lstlisting}

OCaml のデータ型定義は、\texttt{type} キーワードによって行います。
この定義では、\texttt{term} という型を定義しています。
その型コンストラクタは三種類あり、\texttt{EVar} コンストラクタは \texttt{string} 型の値を取り、
\texttt{EApp} コンストラクタは \texttt{term * term} 型の値を取り、
\texttt{EAbs} コンストラクタは \texttt{string * term} 型の値を取ります。
これらのコンストラクタは、それぞれ変数、関数適用、$\lambda$ 抽象に対応しています。
変数は文字列(\texttt{string})型を使っています。

型の中にある \texttt{*} はタプル(組)の型です。\texttt{t1 * t2 * ... * tn} とすると
\texttt{t1}, \texttt{t2}, \dots, \texttt{tn} という型のタプルが作れます。
タプルの値は、その要素の値をコンマで区切って並べることで表現できます。
例えば、\texttt{1}と\texttt{2}のタプルは \texttt{1, 2} のように表記します。

型のデータ型を定義しましょう。型の定義は図\ref{fig:stlc-type}の通りです。
これを定義するとリスト\ref{list:ocaml-def-type}のようになります。

\begin{lstlisting}[caption=型の定義, label=list:ocaml-def-type]
type ty
  = TFun of (ty * ty)
  | TVar of int
\end{lstlisting}

\texttt{type} は OCaml のキーワードなので、\texttt{ty} という名前で型を定義しました。
\texttt{TFun} が関数の型のコンストラクタで、\texttt{TVar} が型変数の型のコンストラクタです。
型変数の実体は \texttt{int} 型(整数型)です。

次に、制約、型環境、代入の型を定義します。

制約は型上の等式なので、二つの型のタプルで表現できます。
制約を定義すると、リスト\ref{list:ocaml-def-tconst}のようになります。

\begin{lstlisting}[caption=制約の定義, label=list:ocaml-def-tconst]
type tconst = ty * ty
\end{lstlisting}

これは、\texttt{ty * ty} という型を、\texttt{tconst} という名前に紐付けています。
型や項の定義と違い、単なる型のエイリアスになっています。

型環境は変数と型のペアの並びのような形を持っていますが、
元々ある変数が後から追加されれば、それを参照することはありません。
つまり、変数から型へのマップ構造を使うことができます。

変数は文字列型を持つので、文字列から型へのマップ構造が必要となります。
文字列のマップは、core ライブラリの \texttt{Core.Std.String.Map} モジュールにあります。
これを使って型環境を定義すると、リスト\ref{list:ocaml-def-type-environment}のようになります。

\begin{lstlisting}[caption=型環境の定義, label=list:ocaml-def-type-environment]
type assump = ty String.Map.t
\end{lstlisting}

\texttt{ty String.Map.t} 型を \texttt{assump} という名前に結び付けています。
\texttt{String.Map.t} という型は、\texttt{String.Map} モジュール内の \texttt{t}
という型を指していますが、これが文字列からのマップの型となっています。
型引数として文字列に対応付ける型を取るので、
\texttt{ty String.Map.t} と書くと文字列から型へのマップの型になります。

代入は型変数から型へのマップ構造です。
型変数は整数型を持つので、整数から型へのマップ構造が必要となります。
整数のマップは、core ライブラリの \texttt{Core.Std.Int.Map} モジュールにあります。

\texttt{Core.Std.String.Map} モジュールと \texttt{Core.Std.Int.Map} モジュールは
それぞれ文字列からのマップと整数値からのマップですが、
その部分の型が違うことを除けば同じ構造、同じ型を持ち、同じように扱えます。

これを使って代入を定義すると、リスト\ref{list:ocaml-def-substitution}のようになります。

\begin{lstlisting}[caption=代入の定義, label=list:ocaml-def-substitution]
type subst = ty Int.Map.t
\end{lstlisting}

型環境の定義と同様に、\texttt{ty Int.Map.t} 型を \texttt{subst} という名前に結び付けています。
この型は、整数値から型へのマップ型です。

\subsection{方程式の生成}

方程式の生成を実装してみましょう。方程式を生成する関数の名前を、\texttt{constraints} とします。
まずは、関数 \texttt{constraints} の型を考えます。

アルゴリズム $\mathcal C$ を見る限りでは、この型は環境と項を取り、等式集合と型を返す型を持ちます。
しかし、それではこの計算の型を十分に表せていません。

まず、この計算の中にフレッシュな型変数を作り出す計算が含まれています。
これは、どのような型変数がまだ使われていないのかという情報を状態として持ち回らなければ実現できません。
型変数は \texttt{int} 型を持つので、0から順に割り当てるものとして、
どこから割り当て可能かを表す \texttt{int} 型の値を状態として持つことにします。
フレッシュな変数を作るごとに、この状態の値は1増えます。

次に、この計算の中には失敗し結果を返さない部分が含まれています。
これも、値があるか無いかのどちらかであることを表す型を使わなければいけません。
OCaml であれば \texttt{option} 型を使うべきでしょう。

以上のことから、\texttt{constraints} は
\texttt{int -> assump -> term -> (int * tconst list * ty) option} 型を持つのが妥当だと考えられます。
引数は割り当て可能な型変数の番号(\texttt{int})、型環境(\texttt{assump})、項(\texttt{term})です。
結果は、遷移後の割り当て可能な型変数(\texttt{int})と等式集合(\texttt{tconst list})と型(\texttt{ty})
のタプルの \texttt{option} 型です。

全体の書き方の方針ですが、
アルゴリズム $\mathcal C$ の定義は項の構造でパターンマッチして書かれています。
\texttt{constraints} も同様に、三番目の引数でパターンマッチして書くことになります。
OCaml では、パターンマッチ分岐のできる無名関数は \texttt{function} キーワードを使って書きます。

\texttt{function} キーワードを使うと \texttt{constraints} 関数全体は、
\begin{lstlisting}
let rec constraints (n : int) (env : assump) :
    term -> (int * tconst list * ty) option =
  function
    | EVar str ->
      ...
    | EApp (term1, term2) ->
      ...
    | EAbs (ident, term) ->
      ...
\end{lstlisting}
のように記述できます。\texttt{let rec} は再帰関数を束縛するためのキーワードです。
一番目、二番目の引数はそれぞれ \texttt{n}、\texttt{env} という名前に束縛されています。
\texttt{...} の部分に、それぞれの項の種類に対する方程式生成の計算が入ります。

残りの \texttt{...} の部分を全て埋めた完全な \texttt{constraints} 関数の定義は、
リスト\ref{list:ocaml-stlc-constraints}の通りです。

\begin{lstlisting}[caption=方程式の生成, label=list:ocaml-stlc-constraints]
let rec constraints (n : int) (env : assump) :
    term -> (int * tconst list * ty) option =
  function
    | EVar str ->
      begin match String.Map.find env str with
        | Some t -> Some (n, [], t)
        | None -> None
      end
    | EApp (term1, term2) ->
      begin match constraints (succ n) env term1 with
        | Some (n1, c1, t1) ->
          begin match constraints n1 env term2 with
            | Some (n2, c2, t2) ->
              let tn = TVar n in
              Some (n2, (t1, TFun (t2, tn)) :: c1 @ c2, tn)
            | None -> None
          end
        | None -> None
      end
    | EAbs (ident, term) ->
      begin
        let tn = TVar n in
        let newenv = String.Map.add ident tn env in
        match constraints (succ n) newenv term with
          | Some (n', c, t) -> Some (n', c, TFun (tn, t))
          | None -> None
      end
\end{lstlisting}

\texttt{option} 型を使っていることと、
フレッシュな型変数の生成のために明示的に状態を持ち回っていることを除けば、
ほとんどアルゴリズム $\mathcal C$ と同等のことをしています。

\texttt{match \dots with \dots} は、\texttt{match} の後の式の値でパターンマッチ分岐する式です。

\texttt{begin} と \texttt{end} は、括弧と同等の意味を持ちます。
\texttt{match} などを括弧で括ると不恰好になってしまうため、
そのような場合に \texttt{begin} と \texttt{end} を使用します。

マップのモジュールの \texttt{find} 関数はマップの値とキーの値を取り、
そのキーに対応する値を取り出す関数です。
ただし、結果は \texttt{option} 型であり、目的の値が無ければ \texttt{None} を返します。
\texttt{add} 関数はマップとキーと値を取り、マップにキーと値を追加してできたマップを返す関数です。

\subsection{制約解消系}

次に、制約解消系を実装しましょう。

二番目の変形規則で使われている型の代入の計算を実装します。

広い範囲に適用できるようにするため、代入(\texttt{subst} 型の値)を取って、
それに対応する型から型への関数を作る形にしましょう。
代入と型を取って、型に代入を適用した型を返すと言っても同じことになります。
以上のことから、代入の関数 \texttt{substitute} は \texttt{subst -> ty -> ty} 型を持ちます。

この方針に基き、\texttt{substitute} 関数をリスト\ref{list:ocaml-stlc-substitute}のように定義します。

\begin{lstlisting}[caption=代入, label=list:ocaml-stlc-substitute]
let rec substitute (s : subst) : ty -> ty =
  function
    | TVar n ->
      begin match Int.Map.find s n with
        | Some t -> t
        | None -> TVar n
      end
    | TFun (tl, tr) -> TFun (substitute s tl, substitute s tr)
\end{lstlisting}

代入を適用する型が型変数であれば、その型変数と型の紐付けが代入に含まれるかを調べます。
含まれていれば対応する型を返し、含まれていなければ型変数をそのまま結果とします。

代入を適用する型が関数の型であれば、その下にある二つの型それぞれに対して代入を適用し、
それらに関しての関数の型を結果とします。

二番目の変形規則では、変数に代入される型がその代入先の変数を含まないことを条件としています。
この条件を満たしているか調べる計算を、\textgt{出現検査 (occurs check)} といいます。

出現検査の関数 \texttt{occurs\_check} は、型変数と型を取って型変数が型に含まれているかを調べるので、
\texttt{int -> ty -> bool} 型を持ちます。

\texttt{occurs\_check} 関数をリスト\ref{list:ocaml-stlc-occurs-check}のように定義します。

\begin{lstlisting}[caption=出現検査, label=list:ocaml-stlc-occurs-check]
let rec occurs_check (n : int) : ty -> bool =
  function
    | TVar n' -> n = n'
    | TFun (tl, tr) -> occurs_check n tl || occurs_check n tr
\end{lstlisting}

この関数は、型の構造に対して帰納的に目的の型変数が含まれていないか探索を行います。

上で定義した二つの関数を使って、単一化の関数を実装します。

単一化の計算をする \texttt{unify} 関数は、
制約集合と代入のペアを徐々に変形させて最終的に代入が得られるかもしくは失敗する計算なので、
\texttt{subst -> tconst list -> subst option} 型を持ちます。

\texttt{unify} 関数をリスト\ref{list:ocaml-stlc-unify}のように定義します。

\begin{lstlisting}[caption=単一化, label=list:ocaml-stlc-unify]
let rec unify (env : subst) : tconst list -> subst option =
  function
    | [] -> Some env
    | (TVar n, TVar n') :: cs when n = n' -> unify env cs
    | (TVar n, t) :: cs | (t, TVar n) :: cs ->
      let sub = substitute (Int.Map.singleton n t) in
      if occurs_check n t
        then None
        else unify
          (Int.Map.add n t (Int.Map.map sub env))
          (List.map cs (fun (l, r) -> sub l, sub r))
    | (TFun (t1l, t1r), TFun (t2l, t2r)) :: cs ->
      unify env ((t1l, t2l) :: (t1r, t2r) :: cs)
\end{lstlisting}

最初のパターンマッチ分岐は、等式集合が空の場合です。
これは一度も書き換えをしなくて良いので、代入をそのまま結果とします。

二番目のパターンマッチ分岐は、一番目の変形規則に対応します。

三番目のパターンマッチ分岐は、二番目の変形規則に対応します。
ただし、出現検査にひっかかる場合は解が存在しないため計算が失敗します。

6行目で使っている \texttt{singleton} 関数は、一つのキーと値のペアしか持たないマップを作ります。
これによって、一つの型変数のみに対する代入を作れます。

四番目のパターンマッチ分岐は、三番目の変形規則に対応します。
型の構造を分解し、等式を二つの等価な等式に分割しています。

元々の一番目の変形規則は同一の型同士の等式に関する規則ですが、
実装としては型変数が同一かどうかしか見ていません。
ただし、同一の型であれば構造も一致するので、全体としては同じ計算になっています。

\subsection{型推論}

ここまでで定義したものを使い、
型推論の関数 \texttt{type\_inference} をリスト\ref{list:ocaml-stlc-type-infer}のように定義できます。

\begin{lstlisting}[caption=型推論, label=list:ocaml-stlc-type-infer]
let type_inference (e : term) : ty option =
  match constraints 0 String.Map.empty e with
    | Some (_, c, t) ->
      begin match unify Int.Map.empty c with
        | Some s -> Some (substitute s t)
        | None -> None
      end
    | None -> None
\end{lstlisting}

\section{まとめ}

第2章では、
型の与えられていない単純型付き $\lambda$ 計算の項の一番広い意味での型を推論するアルゴリズムと、
OCaml によるそのアルゴリズムの実装を示しました。

このアルゴリズムは主に制約生成と単一化という二つの部分から成立しています。
これらの部分に関する理解の上で重要なことは、制約生成は正しい制約を生成して、
単一化は正しくかつ最も広い解を計算しているということです。
それぞれの部分が良い性質を満たしているということから、型推論アルゴリズムの正しさが説明できます。



\chapter{多相型の型推論}


この章では、単純型付き $\lambda$ 計算に多相性を追加した体系の型推論について論じます。

\section{多相の必要性}

$(\lambda x. x) \, 12, (\lambda x . x) \, (\lambda x . x)$ という項の型を考えてみましょう。

ここで、$(e_1, e_2)$ という形で表わされている部分は $e_1$ と $e_2$ の直積(タプル)です。
型 $\tau_1$ と型 $\tau_2$ の直積型を $(\tau_1 \times \tau_2)$ という形で表現します。
直積に関する追加の型付け規則 \textsc{T-Pro} を、図\ref{fig:product-type-judgement}のように定義します。

\begin{figure}[htbp]
  \[
    \infere{T-Pro}{
      \Gamma \vdash (e_1, e_2) : (\tau_1 \times \tau_2)
    }{
      \Gamma \vdash e_1 : \tau_1 &
      \Gamma \vdash e_2 : \tau_2
    }
  \]
  \caption{直積に関する型判定}
  \label{fig:product-type-judgement}
\end{figure}

今考えている項は、整数値の型を $\mathrm{int}$ とすると
$\mathrm{int} \times (\alpha \to \alpha)$ という型で型付けできます。

しかし、この項の中の恒等関数を括り出して $(\lambda f .(f \, 12, f \, f)) \, (\lambda x . x)$
のように書き換えてやると、型付けできなくなってしまいます。
これは、恒等関数の型が元々
$\mathrm{int} \to \mathrm{int}$ と $(\alpha \to \alpha) \to \alpha \to \alpha$ と
$\alpha \to \alpha$ のように全て一致しないことによる問題です。

これらの型はどれも $\tau_1 \to \tau_1$ の形を持っていますが、
それぞれ $\tau_1$ に入る型が違ってしまうために型が合いません。
これでは任意の型に関する関数などが書けず、そのために同じ定義を何度も書かなければなりません。

この章で扱う多相は、この「任意の型」を扱うための多相です。
この多相性を追加した型付き $\lambda$ 計算について考えてみましょう。

\section{let 項}

前節で説明したような多相性を導入することを明示するための新しい項の要素を、
$\lambda$ 項の定義に追加します。

この新しい項の要素を\textbf{let 項 (let term)} と呼び、
$(\letterm{x}{e_1}{e_2})$ のように表記します。
また、この新しく定義される項全体を $\lambda^\mathrm{let}$ 項と呼びます。

$\lambda^{let}$ 項を、図\ref{fig:poly-lambda-term}のように定義します。

\begin{figure}[htbp]
  \begin{align*}
    e & \bnfcce  x                   && \text{(変数)} \\
      & \bnfvert (e \, e)            && \text{(関数適用)} \\
      & \bnfvert (\lambda x . e)     && \text{($\lambda$ 抽象)} \\
      & \bnfvert (\letterm{x}{e}{e}) && \text{(let 項)}
  \end{align*}
  \caption{$\lambda^\mathrm{let}$ 項}
  \label{fig:poly-lambda-term}
\end{figure}

let 項 $(\letterm{x}{e_1}{e_2})$ の簡約の上での意味は $(\lambda x . e_2) \, e_1$ と同一であり、
変数 $x$ が項 $e_1$ に多相的に束縛されるものとします。

$\lambda$ 抽象で束縛されている変数を単相的な変数、let 項で束縛されている変数を多相的な変数と呼びます。

省略に関する追加の規則は以下の通りです。

\begin{enumerate}
  \item 項 $e$ の部分項 $(\letterm{x}{e_1}{e_2})$ は、$e_2$ の左右に対応する括弧があれば、
        その括弧を省略できます。
\end{enumerate}

\begin{note}
ここまでの内容で、省略表記であることが明らかでない let 項の表記は省略されていてはならない。
\end{note}

このような体系を考えることにより、前節で出てきた $(\lambda f .(f \, 12, f \, f)) \, (\lambda x . x)$
という項は、$\letterm{x}{\lambda x . x}{f \, 12, f \, f}$ に書き換えられます。
適切な型付けの体系を与えることで、この項を型付けすることができるようになります。

\section{置換による多相}

\subsection{アルゴリズム}

この節では、$\lambda^\mathrm{let}$ 項に対する最も簡単な型付けの体系と、
その体系に合った適切な型推論アルゴリズムを定義します。

多相を導入する項、即ち let 項を全て簡約してしまえば、
多相性がどこにも現れない値として同値な項が得られるはずです。

let を含む任意の項から、その項中の let 項を全て簡約した項を計算するアルゴリズムが存在します。
このアルゴリズムの名前を $\mathcal R^\mathrm{let}$ とします。

アルゴリズム $\mathcal R^\mathrm{let}$ は図\ref{fig:algorithm-rlet}のように定義できます。

\begin{figure}[htbp]
  \begin{align*}
    \mathcal R^\mathrm{let}(x) &= x \\
    \mathcal R^\mathrm{let}(e_1 e_2) &=
      \mathcal R^\mathrm{let}(e_1) \, \mathcal R^\mathrm{let}(e_2) \\
    \mathcal R^\mathrm{let}(\lambda x . e) &=
      \lambda x . \mathcal R^\mathrm{let}(e) \\
    \mathcal R^\mathrm{let}(\letterm{x}{e_1}{e_2}) &=
      \mathcal R^\mathrm{let}(e_2 [x := \mathcal R^\mathrm{let}(e_1)])
  \end{align*}
  \caption{アルゴリズム $\mathcal R^\mathrm{let}$}
  \label{fig:algorithm-rlet}
\end{figure}

このアルゴリズムは再帰に関して項中の let 項の出現数もしくは項の構造が減少し、
かつ項中の let 項の出現数が増加することは無いため、
項中の let 項の出現数と項の構造の辞書式順序の帰納法を使った正しい帰納的定義になっています。

これによって $\lambda^\mathrm{let}$ 項を $\lambda$ 項に変換し、
2章で示したアルゴリズムによって型推論をすることで、let を含む項の型を推論できます。

ただし、簡約をしてしまうと変数の捕獲が問題になってしまう場合があります。
適切な $\alpha$ 変換をするアルゴリズムを書いても良いのですが、
制約を生成するアルゴリズムに少し変更を加えることによって、
より実装に向いた型推論アルゴリズムを得られます。

$\lambda^\mathrm{let}$ 項には、単相的に束縛される変数と多相的に束縛される変数があります。

型環境の上でこれらを区別するようにしましょう。
単相の変数は型だけを持っておけば良いはずです。一方、多相の変数は型環境と項を持っておき、
変数が使われる度に制約を生成することで適切な制約集合が得られます。
これらを区別するため、前者は持つ型を $\tau$ としたときに $\mathrm{Mono}(\tau)$ と、
後者は型環境と項をそれぞれ $\Gamma$ と $e$ としたときに $\mathrm{Poly}(\Gamma, e)$ と表記します。
\footnote{それぞれ、monomorphic (単相的)と polymorphic (多相的)の略。}

$\lambda^\mathrm{let}$ 項に対する制約生成アルゴリズム $\mathcal C^\mathrm{let}$ を、
図\ref{fig:algorithm-cp}のように定義します。

\begin{figure}[htbp]
  \begin{align*}
    \mathcal C^\mathrm{let}(\Gamma, x) &=
      \begin{cases}
        \mathit{failure}                    & (x \notin \mathit{dom}(\Gamma)) \\
        \tau                                & (\mathrm{Mono}(\tau) = \Gamma(x)) \\
        \mathcal C^\mathrm{let}(\Gamma', e) & (\mathrm{Poly}(\Gamma', e) = \Gamma(x)) \\
      \end{cases} \\
    \mathcal C^\mathrm{let}(\Gamma, e_1 \, e_2) &=
      \begin{array}[t]{l}
        \mathrm{let}
          \begin{array}[t]{l}
            (c_1, \tau_1) = \mathcal C^\mathrm{let}(\Gamma, e_1) \\
            (c_2, \tau_2) = \mathcal C^\mathrm{let}(\Gamma, e_2) \\
            \tau_3 = \mathrm{fresh}
          \end{array} \\
        \mathrm{in} (c_1 \cup c_2 \cup \{\tau_1 = \tau_2 \to \tau_3\}, \tau_3)
      \end{array} \\
    \mathcal C^\mathrm{let}(\Gamma, \lambda x . e) &=
      \begin{array}[t]{l}
        \mathrm{let}
          \begin{array}[t]{l}
            \tau_1 = \mathrm{fresh} \\
            (c, \tau_2) = \mathcal C^\mathrm{let}((\Gamma, x : \mathrm{Mono}(\tau_1)), e)
          \end{array} \\
        \mathrm{in} (c, \tau_1 \to \tau_2)
      \end{array} \\
    \mathcal C^\mathrm{let}(\Gamma, \letterm{x}{e_1}{e_2}) &=
      \mathcal C^\mathrm{let}((\Gamma, x : \mathrm{Poly}(\Gamma, e_1)), e_2)
  \end{align*}
  \caption{方程式の生成アルゴリズム $\mathcal C^\mathrm{let}$}
  \label{fig:algorithm-cp}
\end{figure}

アルゴリズム $\mathcal C$ からの主な変更点は三箇所です。

\begin{itemize}
  \item 項が変数であった場合に、それが単相の変数か多相の変数かによって計算が変わります。

        単相の変数であれば、型環境から得られる型をそのまま返します。
        多相の変数であれば、型環境から得られる型環境と項を用いて制約集合を生成します。

  \item $\lambda$ 抽象に関する制約の生成で、
        型環境に追加される情報が $\mathrm{Mono}(\tau)$ の形になります。

  \item let 項に対する制約の生成が追加されました。

        型環境 $\Gamma$ と $(\letterm{x}{e_1}{e_2})$ という項を取る場合、
        $\Gamma$ に変数 $x$ と $\mathrm{Poly}(\Gamma, e_1)$ の対応付けを追加した型環境を作り、
        その型環境と $e_2$ を使って制約の生成を行います。
\end{itemize}

これと第2章で示した単一化アルゴリズムを用いて、
$\lambda^\mathrm{let}$ 項に対する適切な型推論アルゴリズムを作れます。

\subsection{実装}

多相性を実現するための制約生成アルゴリズムを実装してみましょう。

$\lambda^\mathrm{let}$ 項のデータ型をリスト\ref{list:ocaml-def-ll-term}のように定義します。

\begin{lstlisting}[caption=$\lambda^\mathrm{let}$ 項の定義, label=list:ocaml-def-ll-term]
type term
  = EVar of string
  | EApp of (term * term)
  | EAbs of (string * term)
  | ELet of (string * term * term)
\end{lstlisting}

\texttt{ELet} コンストラクタが let 項のコンストラクタです。

型環境をリスト\ref{list:ocaml-def-poly-type-environment}のように定義します。

\begin{lstlisting}[caption=型環境の定義, label=list:ocaml-def-poly-type-environment]
type assump_elem
    = Monovar of ty
    | Polyvar of (assump_elem String.Map.t * term)
type assump = assump_elem String.Map.t
\end{lstlisting}

\texttt{assump\_elem} 型は型環境の一つの要素を表しています。

\texttt{Monovar} は単相変数用のコンストラクタ、\texttt{Polyvar} は多相変数用のコンストラクタです。

\texttt{assump} 型は型環境を表しています。
第2章での定義と違い、\texttt{assump\_elem} 型を用いていることに注意してください。

\texttt{constraints} 関数をリスト\ref{list:ocaml-poly-constraints}のように定義します。

\begin{lstlisting}[caption=制約の生成, label=list:ocaml-poly-constraints]
let rec constraints (n : int) (env : assump) :
    term -> (int * tconst list * ty) option =
  function
    | EVar str ->
      begin match String.Map.find env str with
        | Some (Monovar t) -> Some (n, [], t)
        | Some (Polyvar (env', term)) -> constraints n env' term
        | None -> None
      end
    | EApp (term1, term2) ->
      begin match constraints (succ n) env term1 with
        | Some (n1, c1, t1) ->
          begin match constraints n1 env term2 with
            | Some (n2, c2, t2) ->
              let tn = TVar n in
              Some (n2, (t1, TFun (t2, tn)) :: c1 @ c2, tn)
            | None -> None
          end
        | None -> None
      end
    | EAbs (ident, term) ->
      begin
        let tn = TVar n in
        let newenv = String.Map.add ident (Monovar tn) env in
        match constraints (succ n) newenv term with
          | Some (n', c, t) -> Some (n', c, TFun (tn, t))
          | None -> None
      end
    | ELet (ident, term1, term2) ->
      let newenv = String.Map.add ident (Polyvar (env, term1)) env in
      constraints n newenv term2
\end{lstlisting}

2章で示した \texttt{constraints} 関数からの変更点はアルゴリズムの変更と同様の三箇所です。
また、型環境の定義が少し違うものになっているため、
変数に関わる部分で \texttt{Monovar} や \texttt{Polyvar} が使われています。

\section{パラメトリック多相}

置き換えによる $\lambda^\mathrm{let}$ 項の型推論は、
この本で目標とする範囲の推論アルゴリズムと完全に一致しています。

しかし、この方法では時間的なコストが大きくなってしまう、
型が一致しないときに問題のある項を適切に示せないなどの問題があり、
多くの言語処理系では別の方法を採用しています。

ここでは、そのより良い方法である\textbf{パラメトリック多相 (parametric polymorphism)}
を用いた型付けの体系と、それに基いた推論アルゴリズムについて説明します。

パラメトリック多相の話に入るために、多相性の関わる項の例を幾つか考えてみます。

まずは、恒等関数 $\lambda x . x$ について考えましょう。
この項の最も広い意味での型は、$\alpha \to \alpha$ です。

$\letterm{i}{\lambda x . x}{\dots}$ のように書くことで、$\dots$ の中で恒等関数 $i$
を多相的に扱うことができます。このとき、その全ての $i$ の型はどれも $\tau_1 \to \tau_1$
の形をしているはずです。つまり、多相的な恒等関数は、任意の型 $\alpha$ に関して $\alpha \to \alpha$
という型を持つと言えます。パラメトリック多相では、このような型を記号 $\forall$ を用いて
$\forall \alpha . \alpha \to \alpha$ のように表記します。

この例のように、項 $e$ が型 $\tau$ で型付けできるとき、
項 $e$ に多相的に束縛された変数 $x$ は幾つかの任意の型に関して型 $\tau$ を持つと言えます。
項 $x$ の型は、変数 $x$ の型の任意の型全てに適切に型を代入することによって得られます。

さて、ここでどの型変数を任意の型として良いのかという問題がありますが、
全ての型変数を任意の型として良いわけではありません。

もう一つ、新しい例として直積型を含む体系における
$\lambda a . \letterm{f}{\lambda b . (a, b)}{(f, f)}$ という項の型を考えてみましょう。

この項は、
$\alpha \to ((\gamma \to (\alpha \times \gamma)) \times (\delta \to (\alpha \times \delta)))$
で型付けでき、かつこれが最も一般的な型です。
これは二つの $f$ の直積を結果としていますが、それぞれの $f$ に相当する型は
$\gamma \to (\alpha \times \gamma)$ と $\delta \to (\alpha \times \delta)$ になっているため、
fの型は $\forall \beta . \beta \to (\alpha \times \beta)$ となっているはずです。
$\beta$ は任意の型となっていますが、$\alpha$ はなっていません。

これは、$\alpha$ が let の外側の型環境に含まれていることによる制限です。
外にもある型変数を勝手に付け替えてしまえば、型の整合性が取れなくなってしまいます。

よって、多相変数の束縛において $\forall$ の後に書かれる型変数は、その多相変数が束縛されている
項やその部分項の型以外の部分(let の外側の型環境)で出現してはいけないことになります。
この他に制約はありませんが、型中に一度も出現していない型変数を任意の型としてしまっても意味が無いので、
多相変数が束縛されている項の型に含まれ、かつその項やその部分項の型以外の部分で全く出現していない
型変数を任意の型として良いと考えるのが良いでしょう。

\subsection{理論}

単相の型のことを型と呼び、多相の($\forall$ の付いた)型のことを型スキームと呼びます。

型を記号 $\tau$ で表し、型スキームを記号 $\sigma$ で表します。
$\lambda^\mathrm{let}$ の型と型スキームを図\ref{fig:polymorphism-type}のように定義します。

\begin{figure}[htbp]
  \begin{align*}
    \tau   & \bnfcce  \alpha \\
           & \bnfvert (\tau \to \tau) \\
    \sigma & \bnfcce  \tau \\
           & \bnfvert \forall \alpha . \sigma
  \end{align*}
  \caption{$\lambda^\mathrm{let}$ の型}
  \label{fig:polymorphism-type}
\end{figure}

型に関しては今までと同様の省略規則を用います。型スキームに関して以下の省略規則を導入します。

\begin{enumerate}
  \item 型スキーム $\forall \alpha . \forall \beta . \dots \forall \gamma . \tau$ は、
        $\forall \alpha \beta \dots \gamma . \tau$ と省略できます。
\end{enumerate}

$\lambda^\mathrm{let}$ の型判定を図\ref{fig:poly-type-judgement}のように定義します。

\begin{figure}[htbp]
  \[
    \infere{T-Var}{
      \Gamma \vdash x : \sigma
    }{
      x : \sigma \in \Gamma
    }
  \]
  \[
    \infere{T-App}{
      \Gamma \vdash e_1 \, e_2 : \tau_2
    }{
      \Gamma \vdash e_1 : \tau_1 \to \tau_2 &
      \Gamma \vdash e_2 : \tau_1
    }
  \]
  \[
    \infere{T-Abs}{
      \Gamma \vdash \lambda x . e : \tau_1 \to \tau_2
    }
    {
      \Gamma, x : \tau_1 \vdash e : \tau_2
    }
  \]
  \[
    \infere{T-Let}{
      \Gamma \vdash \letterm{x}{e_1}{e_2} : \tau
    }{
      \Gamma \vdash e_1 : \sigma &
      \Gamma, x : \sigma \vdash e_2 : \tau
    }
  \]
  \[
    \infere{P-App}{
      \Gamma \vdash e : \sigma [\alpha := \tau]
    }{
      \Gamma \vdash e : \forall \alpha . \sigma
    }
  \]
  \[
    \infere{P-Abs}{
      \Gamma \vdash e : \forall \alpha . \sigma
    }{
      \Gamma \vdash e : \sigma &
      \alpha \notin FTV(\Gamma)
    }
  \]
  \caption{$\lambda^\mathrm{let}$ の型判定}
  \label{fig:poly-type-judgement}
\end{figure}

規則 \textsc{T-Let} は let 項に関する型付け規則です。

規則 \textsc{P-App} は型スキームの任意の型に具体的な型をあてはめ、型を単相にする規則です。

規則 \textsc{P-App} は型環境に含まれない型変数を任意の型とし、型を多相にする規則です。

これらの規則全体を見ると、規則 \textsc{P-App} は規則 \textsc{T-Var} の外側のみに、
規則 \textsc{P-Abs} は規則 \textsc{T-Let} の内側のみに出現すれば
十分に広い意味の型が付くことが分かります。

$\lambda^\mathrm{let}$ の型判定を図\ref{fig:poly-type-judgement2}のように定義し直すことができます。

\begin{figure}[htbp]
  \[
    \infere{T-Var}{
      \Gamma \vdash x : \tau_0 [\alpha := \tau_1, \dots, \beta := \tau_n]
    }{
      x : \forall \alpha \dots \beta . \tau_0 \in \Gamma
    }
  \]
  \[
    \infere{T-App}{
      \Gamma \vdash e_1 \, e_2 : \tau_2
    }{
      \Gamma \vdash e_1 : \tau_1 \to \tau_2 &
      \Gamma \vdash e_2 : \tau_1
    }
  \]
  \[
    \infere{T-Abs}{
      \Gamma \vdash \lambda x . e : \tau_1 \to \tau_2
    }
    {
      \Gamma, x : \tau_1 \vdash e : \tau_2
    }
  \]
  \[
    \infere{T-Let}{
      \Gamma \vdash \letterm{x}{e_1}{e_2} : \tau_2
    }{
      \Gamma \vdash e_1 : \tau_1 &
      \Gamma, x : \forall A . \tau_1 \vdash e_2 : \tau_2 &
      \forall \alpha \in A . \alpha \notin FTV(\Gamma)
    }
  \]
  \caption{$\lambda^\mathrm{let}$ の型判定}
  \label{fig:poly-type-judgement2}
\end{figure}

この型判定規則では、多相性の導入を \textsc{T-Let} 規則に含め、
多相性の除去を \textsc{T-Var} 規則に含むようにしました。

\subsection{アルゴリズム}

新しい $\lambda^\mathrm{let}$ の型判定規則を元に、
$\lambda^\mathrm{let}$ の型推論アルゴリズムを考えてみましょう。

パラメトリック多相を持つ型システムに対する型推論で最も重要なのは、多相型(型スキーム)を作る計算、
即ち多相にして良い型変数の集合の計算だと言えます。

多相にして良い型変数は let 項の外側の型環境に含まれない全ての型変数ですが、
多相にする型に含まれない型変数を多相にしてしまっても意味はありません。
よって、多相にする型に含まれる型変数のうち、
外側の型環境に含まれない全ての型変数を多相にすれば良いことが分かります。

しかし、2章で示したような制約の生成と制約解消を完全に分離した形の推論アルゴリズムでは、
制約の生成の途中で多相にして良い型変数を判定できず、型スキームを作る計算ができません。
この問題を解決するためには、型スキームを作る時点でその部分の制約を解消する必要があります。

制約生成で使う最も一般的な型スキームを作る関数 $\mathit{generalize}$ と型スキームを単相にする関数
$\mathit{instantiate}$ をそれぞれ図\ref{fig:generalize}と\ref{fig:instantiate}のように定義します。

\begin{figure}[htbp]
  \begin{align*}
    \mathit{generalize}(\Gamma, \tau) &= \forall (FTV(\tau) - FTV(\Gamma)) . \tau
  \end{align*}
  \caption{多相型を作る関数 $\mathit{generalize}$}
  \label{fig:generalize}
\end{figure}

\begin{figure}[htbp]
  \begin{align*}
    \mathit{instantiate}(\forall \alpha ... \beta . \tau) &=
      \tau [\alpha := \mathrm{fresh}, \dots, \beta := \mathrm{fresh}]
  \end{align*}
  \caption{多相型を単相にする関数 $\mathit{instantiate}$}
  \label{fig:instantiate}
\end{figure}

$\mathit{generalize}$ では、
型に含まれる型変数のうち型環境に含まれない型変数全てを任意の型とするように型スキームを作っています。

$\mathit{instantiate}$ では、
型スキームの任意の型全てにフレッシュな型を代入しています。

新しい型判定規則を用いた $\lambda^\mathrm{let}$ の制約生成アルゴリズム
$\mathcal C^\mathrm{let'}$ は図\ref{fig:algorithm-c-let2}のように定義できます。

\begin{figure}[htbp]
  \begin{align*}
    \mathcal C^\mathrm{let'}(\Gamma, x) &=
    \begin{cases}
      \mathit{failure}                          & (x \notin \mathit{dom}(\Gamma)) \\
      (\emptyset, \mathit{instantiate}(\sigma)) & (\sigma = \Gamma(x))
    \end{cases} \\
    \mathcal C^\mathrm{let'}(\Gamma, e_1 \, e_2) &=
    \begin{array}[t]{l}
      \mathrm{let}
        \begin{array}[t]{l}
          (c_1, \tau_1) = \mathcal C^\mathrm{let'}(\Gamma, e_1) \\
          (c_2, \tau_2) = \mathcal C^\mathrm{let'}(\Gamma, e_2) \\
          \tau_3 = \mathrm{fresh}
        \end{array} \\
      \mathrm{in} (c_1 \cup c_2 \cup \{\tau_1 = \tau_2 \to \tau_3\}, \tau_3)
    \end{array} \\
    \mathcal C^\mathrm{let'}(\Gamma, \lambda x . e) &=
    \begin{array}[t]{l}
      \mathrm{let}
        \begin{array}[t]{l}
          \tau_1 = \mathrm{fresh} \\
          (c, \tau_2) = \mathcal C^\mathrm{let'}((\Gamma, x : \tau_1), e)
        \end{array} \\
      \mathrm{in} (c, \tau_1 \to \tau_2)
    \end{array} \\
    \mathcal C^\mathrm{let'}(\Gamma, \letterm{x}{e_1}{e_2}) &=
    \begin{array}[t]{l}
      \mathrm{let}
        \begin{array}[t]{l}
          (c_1, \tau_1) = \mathcal C^\mathrm{let'}(\Gamma, e_1) \\
          \tau_1'       = \tau_1 \, \mathcal U(c_1) \\
          (c_2, \tau_2) = \mathcal C^\mathrm{let'}((\Gamma, x : \mathit{generalize(\Gamma, \tau_1')}) , e_2)
        \end{array} \\
      \mathrm{in} (c_1 \cup c_2, \tau_2)
    \end{array}
  \end{align*}
  \caption{方程式の生成アルゴリズム $\mathcal C^\mathrm{let'}$}
  \label{fig:algorithm-c-let2}
\end{figure}

このアルゴリズムでは、項が let 項であるときに多相になる項の制約生成で得られた制約を
アルゴリズム $\mathcal U$ によって解き、その結果を用いて多相型を作っています。

\section{$\mathcal W$}

\subsection{アルゴリズム}

アルゴリズム $\mathcal C^\mathrm{let'}$ では制約の生成の途中で一部分だけ制約を解くようにしましたが、
そうしてしまうのであれば制約集合を持っておく必要は無く、
制約はその場で解いてしまえば良いことになります。

そのようにして得られる $\lambda^\mathrm{let}$ の型推論アルゴリズムが $\mathcal W$ です。
アルゴリズム $\mathcal W$ は型環境と項のペアを取り、代入と型のペアを返します。
アルゴリズム $\mathcal W$ が返す代入は、
今までの制約生成アルゴリズムが返していた制約を解いたものに相当します。

アルゴリズム $\mathcal W$ の定義を図\ref{fig:algorithm-w}に示します。

\begin{figure}[htbp]
  \begin{align*}
    \mathcal W(\Gamma, x) &=
    \begin{cases}
      \mathit{failure} & (x \notin \mathit{dom}(\Gamma)) \\
      (\emptyset, \mathit{instantiate}(\sigma)) & (\sigma = \Gamma(x))
    \end{cases} \\
    \mathcal W(\Gamma, e_1 \, e_2) &=
    \begin{array}[t]{l}
      \mathrm{let}
        \begin{array}[t]{l}
          (s_1, \tau_1) = \mathcal W(\Gamma, e_1) \\
          (s_2, \tau_2) = \mathcal W(\Gamma \, s_1, e_2) \\
          \tau_3 = \mathrm{fresh} \\
          s_3 = \mathcal U(\{\tau_1 \, s_2 = \tau_2 \to \tau_3\})
        \end{array} \\
      \mathrm{in} (s_1 \cup s_2 \cup s_3, \tau_3 \, s_3)
    \end{array} \\
    \mathcal W(\Gamma, \lambda x. e) &=
    \begin{array}[t]{l}
      \mathrm{let}
        \begin{array}[t]{l}
          \tau_2 = \mathrm{fresh} \\
          (s_1, \tau_1) = \mathcal W((\Gamma, x : \tau_2), e) \\
        \end{array} \\
      \mathrm{in} (s_1, \tau_2 \, s_1 \to \tau_1)
    \end{array} \\
    \mathcal W(\Gamma, \letterm{x}{e_1}{e_2}) &=
    \begin{array}[t]{l}
      \mathrm{let}
        \begin{array}[t]{l}
          (s_1, \tau_1) = \mathcal W(\Gamma, e_1) \\
          (s_2, \tau_2) = \mathcal W((\Gamma \, s_1, x : \mathit{generalize}(\Gamma \, s_1, \tau_1)), e_2) \\
        \end{array} \\
      \mathrm{in} (s_1 \cup s_2, \tau_2)
    \end{array}
  \end{align*}
  \caption{型推論アルゴリズム $\mathcal W$}
  \label{fig:algorithm-w}
\end{figure}



\subsection{実装}



\section{まとめ}





\end{document}

