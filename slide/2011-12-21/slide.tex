\documentclass[cjk, 14pt]{beamer}

\usepackage{amsmath}
\usepackage{amssymb}
\usepackage{listings}
\usepackage{ascmac}
\usepackage{framed}
\usepackage{tikz}

\usetheme{Copenhagen}
\usecolortheme{seahorse}
\useinnertheme{rounded}
\useoutertheme{shadow}

\setbeamercovered{transparent}

\lstset{
  basicstyle=\ttfamily,
  basicstyle=\color{white}\ttfamily\scriptsize,
  backgroundcolor=\color{black},
  columns=[l]{fullflexible}
}

\title{Algolithm $\mathcal W$ 入門}
\subtitle{分かりやすい型推論入門本}
\author{坂口和彦}
\institute{筑波大学 情報学群 情報科学類 B1}
\date{2011/12/21}


\begin{document}

\begin{frame}[empty]

  \titlepage

\end{frame}

\begin{frame}{テーマ : Algorithm $\mathcal W$ 入門}

 \begin{itemize}
  \item Hindley-Milner の型推論アルゴリズム(Algorithm $\mathcal W$)の実装
  \item 型推論の解説本
 \end{itemize}
 を作る

\end{frame}

\begin{frame}{なぜ型推論?}

 \begin{itemize}
  \item 型推論の機能を持つ言語を使う上で役に立つ
  \item 言語処理系を実装する上で役に立つ
  \item 学ぶ対象としてとても面白い
  \item 好きだから
 \end{itemize}

\end{frame}

\begin{frame}{成果物}

 \begin{itemize}
  \item \texttt{http://github.com/pi8027/typeinfer}
  \begin{itemize}
   \item OCaml による何種類かの型推論の実装
   \item Algorithm $\mathcal W$ 入門
  \end{itemize}
 \end{itemize}

 % ここで本の実物を見せる

\end{frame}

\begin{frame}{実装}

 \begin{itemize}
  \item 単純型付き$\lambda$計算の型推論 (\texttt{stlc})
  \item 置換による多相型の型推論 (\texttt{poly})
  \item パラメトリック多相型の型推論 (\texttt{parametric-poly1})
  \item Algorithm $\mathcal W$ (\texttt{parametric-poly2})
  \item Algorithm $\mathcal W$ (\texttt{parametric-poly3})
 \end{itemize}

\end{frame}

\begin{frame}{書籍}

 (ここに章の(?)依存関係グラフを入れる)

\end{frame}

\begin{frame}[plain]

 ご清聴ありがとうございました。

\end{frame}

\end{document}
