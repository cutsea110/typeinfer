\documentclass[cjk, 17pt]{beamer}

\usepackage{amsmath}
\usepackage{amssymb}
\usepackage{listings}
\usepackage{ascmac}
\usepackage{framed}
\usepackage{tikz}

\usetheme{Copenhagen}
\usecolortheme{seahorse}
\useinnertheme{rounded}
\useoutertheme{shadow}

\setbeamercovered{transparent}

\title{Algolithm $\mathcal W$ 入門}
\subtitle{2011年度 情報特別演習I}
\author{坂口和彦}
\institute{筑波大学 情報学群 情報科学類 B1}
\date{2011/12/21}


\begin{document}

\begin{frame}[empty]

  \titlepage

\end{frame}

\begin{frame}{テーマ : Algorithm $\mathcal W$ 入門}

 \begin{itemize}
  \item Hindley-Milner の型推論アルゴリズム(Algorithm $\mathcal W$)の実装
  \item 型推論の解説本
 \end{itemize}
 を作る

\end{frame}

\begin{frame}{型推論とは}

 型情報の欠けたプログラムから、それが持つ型を推論する計算のこと。

 例: $\lambda f \, g \, x . f \, x \, (g \, x)$ という項は、
 $(\alpha \to \beta \to \gamma) \to (\alpha \to \beta) \to \alpha \to \gamma$
 という型を持つ。

\end{frame}

\begin{frame}{なぜ型推論?}

 \begin{itemize}
  \item 型推論の機能を持つ言語を使う上で役に立つ
  \item 言語処理系を実装する上で役に立つ
  \item 学ぶ対象としてとても面白い
  \item 好きだから
 \end{itemize}

\end{frame}

\begin{frame}{成果物}

 \begin{itemize}
  \item http://github.com/pi8027/typeinfer
  \begin{itemize}
   \item OCaml による何種類かの型推論の実装
   \item Algorithm $\mathcal W$ 入門
  \end{itemize}
 \end{itemize}

\end{frame}

\begin{frame}{実装}

 \begin{itemize}
  \item 単純型付き$\lambda$計算の型推論
  \item 置換による多相型の型推論
  \item パラメトリック多相型の型推論
  \item Algorithm $\mathcal W$
 \end{itemize}

\end{frame}

\begin{frame}{解説本}

\includegraphics[width=110mm]{graph.ps}

\end{frame}

\begin{frame}{まとめ}

  \begin{itemize}
    \item 主に二つの目標を設定し、それぞれを達成しました
    \begin{itemize}
      \item 型推論の実装
      \item 解説本の製作
    \end{itemize}
    \item 評価
    \begin{itemize}
      \item 内容は最初予定していたものよりも少ない
      \item 分かりやすさの面では、目標以上のことを達成できた
    \end{itemize}
  \end{itemize}

\end{frame}

\end{document}
